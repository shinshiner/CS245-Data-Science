\documentclass[12pt,a4paper]{article}
\usepackage{ctex}
\usepackage{amsmath,amscd,amsbsy,amssymb,latexsym,url,bm,amsthm}
\usepackage{epsfig,graphicx,subfigure}
\usepackage{enumitem,balance}
\usepackage{wrapfig}
\usepackage{mathrsfs, euscript}
\usepackage[usenames]{xcolor}
\usepackage{hyperref}
\usepackage[vlined,ruled,commentsnumbered,linesnumbered]{algorithm2e}
\usepackage{float}
\usepackage{array}
\usepackage{diagbox}
\usepackage{color}
\usepackage{indentfirst}
\usepackage{fancyhdr}
\usepackage{gensymb}
\usepackage{geometry}
\usepackage{setspace}
\usepackage{aurical}
\usepackage{times}
\usepackage{caption}
\usepackage{fontspec}
\usepackage{booktabs}
\setmainfont{Times New Roman}

\newtheorem{theorem}{Theorem}[section]
\newtheorem{lemma}[theorem]{Lemma}
\newtheorem{proposition}[theorem]{Proposition}
\newtheorem{corollary}[theorem]{Corollary}
\newtheorem{exercise}{Exercise}[section]
\newtheorem*{solution}{Solution}
\theoremstyle{definition}


\renewcommand{\thefootnote}{\fnsymbol{footnote}}

\newcommand{\postscript}[2]
 {\setlength{\epsfxsize}{#2\hsize}
  \centerline{\epsfbox{#1}}}

\renewcommand{\baselinestretch}{1.0}

\setlength{\oddsidemargin}{-0.365in}
\setlength{\evensidemargin}{-0.365in}
\setlength{\topmargin}{-0.3in}
\setlength{\headheight}{0in}
\setlength{\headsep}{0in}
\setlength{\textheight}{10.1in}
\setlength{\textwidth}{7in}
\makeatletter \renewenvironment{proof}[1][Proof] {\par\pushQED{\qed}\normalfont\topsep6\p@\@plus6\p@\relax\trivlist\item[\hskip\labelsep\bfseries#1\@addpunct{.}]\ignorespaces}{\popQED\endtrivlist\@endpefalse} \makeatother
\makeatletter
\renewenvironment{solution}[1][Solution] {\par\pushQED{\qed}\normalfont\topsep6\p@\@plus6\p@\relax\trivlist\item[\hskip\labelsep\bfseries#1\@addpunct{.}]\ignorespaces}{\popQED\endtrivlist\@endpefalse} \makeatother



\begin{document}
\noindent
%==========================================================
\noindent\framebox[\linewidth]{\shortstack[c]{
\Large{\textbf{Report on}}\vspace{1mm}\\ 
\Large{\emph{Visualization and exploration of Adult Dataset}}\vspace{1mm}\\
CS245, Data Science Foundation, Chaojun Lu, Autumn 2017 \vspace{1mm} \\
叶泽林 515030910468}}
\vspace{-1.5\baselineskip}
\section{Introduction} % or Problem Description ?

Currently, data science is becoming ubiquitous in our society and showing its essentiality in many domains (e.g. medical industry~\cite{medical1, medical2}, finance~\cite{finance}, social media~\cite{media1, media2}). With the rapid evolution and wide applications of data science, a series of efficient packages have been constantly developed~\cite{numpy, pandas, matplotlib, sklearn}. Utilizing these packages skillfully is of great importance nowadays. In this project, I tend to conduct an exploration over the \textit{Adult} dataset and visualize the results with some of these packages.

\vspace{-1\baselineskip}
\section{Approaches}

The \textit{Adult} dataset is in the format of .CSV with 16281 lines, each line contains some basic information (age, job, gender and etc) of an adults. I explore and extract the information hidden in the data with the following steps:
\begin{enumerate}
\item Read the data with \textit{pandas}~\cite{pandas} and reconstruct it as \textit{DataFrame} type.

\item Select a target item (e.g. work time per week) to conduct analysis.

\item Select two items to explore the relation between them.

\item Visualize all analysis results with \textit{matplotlib}~\cite{matplotlib} and \textit{pandas}.
\end{enumerate}


\section{Experiments}
\subsection{The Distribution of Single Item}

\subsection{The Relation between Two Items}


\section{Conclusion and Discussion}
In this project, I carry out the visualization and exploration of \textit{Adult} dataset and discover some interesting phenomenons reflected from it.

According to the results of above analysis, the distribution of each item has some kinds of ralation with others. Thus, I think I can train a model to predict target items from other items in my future learning process.

Ultimately, I tend to express my sincere thanks to Professor Chaojun Lu for his patient explanation and guidance in lectures! Thank you!
{\small

\renewcommand{\refname}{References}
\bibliographystyle{ieeetr}
\bibliography{bio}
%========================================================================
\end{document}