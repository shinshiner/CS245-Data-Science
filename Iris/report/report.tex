\documentclass[12pt,a4paper]{article}
\usepackage{ctex}
\usepackage{amsmath,amscd,amsbsy,amssymb,latexsym,url,bm,amsthm}
\usepackage{epsfig,graphicx,subfigure}
\usepackage{enumitem,balance}
\usepackage{wrapfig}
\usepackage{mathrsfs, euscript}
\usepackage[usenames]{xcolor}
\usepackage{hyperref}
\usepackage[vlined,ruled,commentsnumbered,linesnumbered]{algorithm2e}
\usepackage{float}
\usepackage{array}
\usepackage{diagbox}
\usepackage{color}
\usepackage{indentfirst}
\usepackage{fancyhdr}
\usepackage{gensymb}
\usepackage{geometry}
\usepackage{setspace}
\usepackage{aurical}
\usepackage{times}
\usepackage{caption}
\usepackage{fontspec}
\usepackage{booktabs}
\usepackage{listings}
\usepackage{xcolor}
\setmainfont{Times New Roman}

\newtheorem{theorem}{Theorem}[section]
\newtheorem{lemma}[theorem]{Lemma}
\newtheorem{proposition}[theorem]{Proposition}
\newtheorem{corollary}[theorem]{Corollary}
\newtheorem{exercise}{Exercise}[section]
\newtheorem*{solution}{Solution}
\theoremstyle{definition}

\newcommand{\postscript}[2]
 {\setlength{\epsfxsize}{#2\hsize}
  \centerline{\epsfbox{#1}}}

\renewcommand{\baselinestretch}{1.0}

\setlength{\oddsidemargin}{-0.365in}
\setlength{\evensidemargin}{-0.365in}
\setlength{\topmargin}{-0.3in}
\setlength{\headheight}{0in}
\setlength{\headsep}{0in}
\setlength{\textheight}{10.1in}
\setlength{\textwidth}{7in}
\makeatletter \renewenvironment{proof}[1][Proof] {\par\pushQED{\qed}\normalfont\topsep6\p@\@plus6\p@\relax\trivlist\item[\hskip\labelsep\bfseries#1\@addpunct{.}]\ignorespaces}{\popQED\endtrivlist\@endpefalse} \makeatother
\makeatletter
\renewenvironment{solution}[1][Solution] {\par\pushQED{\qed}\normalfont\topsep6\p@\@plus6\p@\relax\trivlist\item[\hskip\labelsep\bfseries#1\@addpunct{.}]\ignorespaces}{\popQED\endtrivlist\@endpefalse} \makeatother

\begin{document}
\noindent
%==========================================================
\noindent\framebox[\linewidth]{\shortstack[c]{
\Large{\emph{探索}Iris\emph{(鸢尾花)数据集}}\vspace{1mm}\\
CS245 \quad 数据科学基础 \quad 陆朝俊 \vspace{1mm} \\
叶泽林 515030910468}}
\vspace{-1.5\baselineskip}

\section{问题描述}

统计学主要研究事物的数量方面,目的是探索数据集的数量特征。而统计学中的描述统计学借助图表或概括性的数值将数据集展示为清晰可理解的形式。在之前的研究中,已经使用了图表对Adults数据集进行了探索和可视化展示,这次研究的主要目标是探索Iris数据集,并通过一些概括性的数值对其进行展示,详细目标如下:

\begin{enumerate}
	\item 探索Iris数据集的基本属性(如数据集总体描述、数据维数、特征名称等);
	
	\item 探索各特征的最小值、最大值、均值、中位数、标准差;
	
	\item 探索各特征之间,以及特征与目标之间的相关性(相关系数)。
\end{enumerate}

\section{解决方案}

\subsection{数据集及其基本属性的获取}

Iris数据集也称鸢尾花卉数据集,是一个常用的多重变量分析数据集,已集成在Python的scikit-learn模块下。安装好相应模块后,只需在命令行输入以下命令即可读取Iris数据集:

\begin{lstlisting}[language=Python,
numbers=left,
keywordstyle=\color{blue!70},
frame=shadowbox,
breaklines=True]
from sklearn import datasets
iris = datasets.load_iris()
\end{lstlisting}

数据集获取完毕后,只需要调用数据集对象的一系列方法即可获得其基本属性,详细结果可参见 \ref{sec:iris_basic} 小节。

\subsection{各特征的分析}
\label{sec:app_attri}

通过第一步的分析,我已经确定Iris所有观测实例的特征数据为numpy.ndarray类型,而对于数值分析(如求平均值、方差等),numpy模块已经提供了简便且高效的函数,因此这一部分的分析可以分为两步:

\begin{enumerate}
	\item 从Iris对象中提取出所有实例的某个特征;
	
	\item  使用numpy模块中的函数求得该特征的最小值、最大值、均值等。
\end{enumerate}

\subsection{相关性分析}

这一部分的分析方法和 \ref{sec:app_attri} 小节基本相同,仍然是使用numpy模块中的函数求得各特征之间、以及特征与目标之间的相关系数,从而分析其相关性,详细结果可参见 \ref{sec:iris_coff} 小节。

\section{结果展示}

\subsection{Iris数据集基本属性}
\label{sec:iris_basic}
\begin{table}[H]
	\renewcommand\arraystretch{1.35}
	\caption{Iris数据集基本属性}
	\label{tab:iris_basic}
	\centering
	
	\begin{tabular}{c|c}
		\centering
		属性 & 值 \\
		\hline
		实例的数据个数 & 150 \\
		实例的数据类型 & numpy.ndarray \\
		实例的数据维数 & (150, 4) \\
		特征名 & 萼片长度,萼片宽度,花瓣长度,花瓣宽度 (单位均为cm) \\
		实例的类别值 & 0,1,2 \\
		实例的类别维数 & (150,) \\
		类别名称 & setosa(山鸢尾), versicolor(杂色鸢尾), virginica(维吉尼亚鸢尾) \\
	\end{tabular}
\end{table}

\subsection{各特征的数值描述}

详细结果可参见表 \ref{tab:iris_att} (考虑到有些统计量无单位,因此在表格中不把单位显式地表示出,各特征单位均为cm)。

\begin{table}[H]
	\renewcommand\arraystretch{1.35}
	\caption{Iris数据集各特征的数值描述}
	\label{tab:iris_att}
	\centering
	
	\begin{tabular}{c|cccc}
		\centering
		 & 萼片长度 & 萼片宽度 & 花瓣长度 & 花瓣宽度 \\
		\hline
		最小值 & 4.3 & 2.0 & 1.0 & 0.1 \\
		最大值 & 7.9 & 4.4 & 6.9 & 2.5 \\
		均值 & 5.843 & 3.054 & 3.759 & 1.120 \\
		中位数 & 5.80 & 3.00 & 4.35 & 1.30 \\
		标准差 & 0.825 & 0.432 & 1.759 & 0.761 \\
		方差 & 0.681 & 0.187 & 3.092 & 0.579 \\
		极差 & 3.6 & 2.4 & 5.9 & 2.4 \\
		下四分位数 & 5.1 & 2.8 & 1.6 & 0.3 \\
		上四分位数 & 6.4 & 3.3 & 5.1 & 1.8 \\
	\end{tabular}
\end{table}

\subsection{各特征及特征与目标之间的相关性}
\label{sec:iris_coff}

\begin{table}[H]
	\renewcommand\arraystretch{1.35}
	\caption{Iris数据集各特征及特征与目标之间的相关性}
	\label{tab:iris_co}
	\centering
	
	\begin{tabular}{c|cccc}
		\centering
		 & 萼片长度 & 萼片宽度 & 花瓣长度 & 花瓣宽度 \\
		\hline
		萼片长度 & 1.000 & -0.109 & 0.872 & 0.818 \\
		萼片宽度 & -0.109 & 1.000 & -0.421 & 0.357 \\
		花瓣长度 & 0.872 & -0.421 & 1.000 & 0.963 \\
		花瓣宽度 & 0.818 & -0.357 & 0.963 & 1.000 \\
		目标 & 0.783 & -0.420 & 0.949 & 0.956 \\
	\end{tabular}
\end{table}

\begin{appendix}
	\section{附录}
	\subsection{Iris数据集的其余基本属性}
	
	\subsubsection{Iris数据集的基本描述}
		\begin{lstlisting}[
		frame=shadowbox,
		breaklines=True,
		title=Iris数据集基本描述]
Notes
-----
Data Set Characteristics:
    :Number of Instances: 150 (50 in each of three classes)
    :Number of Attributes: 4 numeric, predictive attributes and the class
    :Attribute Information:
        - sepal length in cm
        - sepal width in cm
        - petal length in cm
        - petal width in cm
        - class:
                - Iris-Setosa
                - Iris-Versicolour
                - Iris-Virginica
    :Summary Statistics:

    ============== ==== ==== ======= ===== ====================
                    Min  Max   Mean    SD   Class Correlation
    ============== ==== ==== ======= ===== ====================
    sepal length:   4.3  7.9   5.84   0.83    0.7826
    sepal width:    2.0  4.4   3.05   0.43   -0.4194
    petal length:   1.0  6.9   3.76   1.76    0.9490  (high!)
    petal width:    0.1  2.5   1.20  0.76     0.9565  (high!)
    ============== ==== ==== ======= ===== ====================

    :Missing Attribute Values: None
    :Class Distribution: 33.3% for each of 3 classes.
    :Creator: R.A. Fisher
    :Donor: Michael Marshall (MARSHALL%PLU@io.arc.nasa.gov)
    :Date: July, 1988

This is a copy of UCI ML iris datasets.
http://archive.ics.uci.edu/ml/datasets/Iris

The famous Iris database, first used by Sir R.A Fisher

This is perhaps the best known database to be found in the
pattern recognition literature.  Fisher's paper is a classic in the field and
is referenced frequently to this day.  (See Duda & Hart, for example.)  The
data set contains 3 classes of 50 instances each, where each class refers to a
type of iris plant.  One class is linearly separable from the other 2; the
latter are NOT linearly separable from each other.

References
----------
   - Fisher,R.A. "The use of multiple measurements in taxonomic problems"
     Annual Eugenics, 7, Part II, 179-188 (1936); also in "Contributions to
     Mathematical Statistics" (John Wiley, NY, 1950).
   - Duda,R.O., & Hart,P.E. (1973) Pattern Classification and Scene Analysis.
     (Q327.D83) John Wiley & Sons.  ISBN 0-471-22361-1.  See page 218.
   - Dasarathy, B.V. (1980) "Nosing Around the Neighborhood: A New System
     Structure and Classification Rule for Recognition in Partially Exposed
     Environments".  IEEE Transactions on Pattern Analysis and Machine
     Intelligence, Vol. PAMI-2, No. 1, 67-71.
   - Gates, G.W. (1972) "The Reduced Nearest Neighbor Rule".  IEEE Transactions
     on Information Theory, May 1972, 431-433.
   - See also: 1988 MLC Proceedings, 54-64.  Cheeseman et al"s AUTOCLASS II
     conceptual clustering system finds 3 classes in the data.
   - Many, many more ...
		\end{lstlisting}
		
	\subsection{主要代码}
		\begin{lstlisting}[language=Python,
		numbers=left,
		keywordstyle=\color{blue!70},
		commentstyle=\color{green!99!blue!99},
		frame=shadowbox,
		breaklines=True,
		title=Iris数据集探索代码]
from sklearn import datasets
import stats
import numpy as np
import xlwt

class IrisAnalyzer(object):
    def __init__(self):
        # iris data
        self.iris = datasets.load_iris()
        self.data = self.iris.data
        self.target = self.iris.target
        print(self.iris.DESCR)

        # save results as excel tables
        self.wbk = xlwt.Workbook()
        self.char_sheet = self.wbk.add_sheet('各特征分析')
        self.corr_sheet = self.wbk.add_sheet('相关分析')
        self.metrics = {'max': np.max, 'min': np.min, 'avg': np.mean,
                    'median': np.median, 'ptp': np.ptp,
                    'std': np.std, 'var': np.var, 'q1': stats.quantile,
                    'q3': stats.quantile}

        # initialize the excel tables
        for i, name in enumerate(self.iris.feature_names):
            self.char_sheet.write(0, i + 1, name)
        for i, m in enumerate(self.metrics.keys()):
            self.char_sheet.write(i + 1, 0, m)
        for i, name in enumerate(self.iris.feature_names):
            self.corr_sheet.write(0, i + 1, name)
        for i, m in enumerate(self.iris.feature_names):
            self.corr_sheet.write(i + 1, 0, m)
        self.corr_sheet.write(i + 2, 0, 'target')

        print('----------analyzer started------------')

    def run(self):
        # starting to analyze Iris dataset
        self.basic_att()
        self.char_analysis()
        self.corr_analysis()

        self.wbk.save('results.xls')
        print('----------results saved------------')

    def basic_att(self):
        # get basic attributes of Iris dataset
        with open('basic_att.txt', 'w') as f:
            f.write(self.iris.DESCR + '\n\n')

            f.write('Type of data: ' + str(type(self.data)) + '\n\n')
            f.write('Shape of data: ' + str(self.data.shape) + '\n\n')
            f.write('Feature names: ' + str(self.iris.feature_names) + '\n\n')

            f.write('Target: ' + str(self.target) + '\n\n')
            f.write('Type of target: ' + str(type(self.target)) + '\n\n')
            f.write('Shape of target: ' + str(self.target.shape) + '\n\n')
            f.write('Target names: ' + str(self.iris.target_names) + '\n\n')

    def char_analysis(self):
        # analyzing each of characters
        for i in range(self.data.shape[1]):
            char_data = self.data[:,i]
            for j, m in enumerate(self.metrics.keys()):
                if not m in ['q1', 'q3']:
                    self.char_sheet.write(j + 1, i + 1, self.metrics[m](char_data))
                elif m == 'q1':
                    self.char_sheet.write(j + 1, i + 1, self.metrics[m](char_data, p=0.25))
                elif m == 'q3':
                    self.char_sheet.write(j + 1, i + 1, self.metrics[m](char_data, p=0.75))

    def corr_analysis(self):
        # analyzing the correlationships
        for i in range(self.data.shape[1]):
            char_data = self.data[:, i]

            # corr among characters
            for j in range(self.data.shape[1]):
                self.corr_sheet.write(j + 1, i + 1, np.corrcoef(char_data, self.data[:, j])[0][1])

            # corr between characters and target
            self.corr_sheet.write(j + 2, i + 1, np.corrcoef(char_data, self.target)[0][1])

def debug():
    # just used to debug, ignore it
    pass

if __name__ == '__main__':
    ia = IrisAnalyzer()
    ia.run()
		\end{lstlisting}
	
\end{appendix}

%========================================================================
\end{document}