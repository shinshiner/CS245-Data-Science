\documentclass[12pt,a4paper]{article}
\usepackage{ctex}
\usepackage{amsmath,amscd,amsbsy,amssymb,latexsym,url,bm,amsthm}
\usepackage{epsfig,graphicx,subfigure}
\usepackage{enumitem,balance}
\usepackage{wrapfig}
\usepackage{mathrsfs, euscript}
\usepackage[usenames]{xcolor}
\usepackage{hyperref}
\usepackage[vlined,ruled,commentsnumbered,linesnumbered]{algorithm2e}
\usepackage{float}
\usepackage{array}
\usepackage{diagbox}
\usepackage{color}
\usepackage{indentfirst}
\usepackage{fancyhdr}
\usepackage{gensymb}
\usepackage{geometry}
\usepackage{setspace}
\usepackage{aurical}
\usepackage{times}
\usepackage{caption}
\usepackage{fontspec}
\usepackage{booktabs}
\usepackage{listings}
\usepackage{xcolor}
\setmainfont{Times New Roman}

\newtheorem{theorem}{Theorem}[section]
\newtheorem{lemma}[theorem]{Lemma}
\newtheorem{proposition}[theorem]{Proposition}
\newtheorem{corollary}[theorem]{Corollary}
\newtheorem{exercise}{Exercise}[section]
\newtheorem*{solution}{Solution}
\theoremstyle{definition}

\newcommand{\tabincell}[2]{\begin{tabular}{@{}#1@{}}#2\end{tabular}}
\newcommand{\postscript}[2]
 {\setlength{\epsfxsize}{#2\hsize}
  \centerline{\epsfbox{#1}}}

\renewcommand{\baselinestretch}{1.05}

\setlength{\oddsidemargin}{-0.365in}
\setlength{\evensidemargin}{-0.365in}
\setlength{\topmargin}{-0.3in}
\setlength{\headheight}{0in}
\setlength{\headsep}{0in}
\setlength{\textheight}{10.1in}
\setlength{\textwidth}{7in}
\makeatletter \renewenvironment{proof}[1][Proof] {\par\pushQED{\qed}\normalfont\topsep6\p@\@plus6\p@\relax\trivlist\item[\hskip\labelsep\bfseries#1\@addpunct{.}]\ignorespaces}{\popQED\endtrivlist\@endpefalse} \makeatother
\makeatletter
\renewenvironment{solution}[1][Solution] {\par\pushQED{\qed}\normalfont\topsep6\p@\@plus6\p@\relax\trivlist\item[\hskip\labelsep\bfseries#1\@addpunct{.}]\ignorespaces}{\popQED\endtrivlist\@endpefalse} \makeatother

\begin{document}
\noindent
%==========================================================
\noindent\framebox[\linewidth]{\shortstack[c]{
\Large{\emph{关联规则分析与}Apriori\emph{算法}}\vspace{1mm}\\
CS245 \quad 数据科学基础 \quad 陆朝俊 \vspace{1mm} \\
叶泽林 515030910468}}

\section{问题描述}

关联规则分析是数据挖掘中活跃的研究方法之一,其目的是在一个数据集中找出各项之间的关联关系(这种关系一般没有在数据中直接表示出来)。Apriori是关联规则分析中最常用也是最经典的挖掘频繁项集的算法,在本次作业中,我将实现Apriori算法,从交易数据集中发现频繁项集,并生成相应的关联规则。

\section{解决方案\protect\footnote{本次作业的主要代码实现可参见附录 \ref{apd:code}}}

\subsection{数据集的准备}

为验证Apriori算法实现的正确性,我根据一定的规则生成了一个交易数据集,数据集的每个实例代表一条交易记录,共100个实例。我将商品分为食品(bread、milk、apple、orange、beer)电器(TV、PC、phone、fridge、ele\_oven)和工具(scissors、stapler、plate、knife、glue)三类,各按照一定的出现及组合概率生成相应的商品交易记录,具体的参数设定可参见附录 \ref{apd:dataset}。

\vspace{0.01\linewidth}
我认为对于交易情况的分析,仅使用模拟生成的数据集难以取得真实的结果,因此我额外在Groceries数据集上执行了Apriori算法。Groceries数据集是内置于R语言的关联分析数据集,来源于某杂货店一个月的真实交易记录,包含9835条交易记录及169种商品。我将其从R语言包中提取出并重组为.csv格式,再使用Apriori算法进行分析。

\subsection{Apriori算法}

Apriori算法是最经典的挖掘频繁项集的算法,实现了在大数据集上可行的关联规则提取,其核心思想是\textbf{通过连接产生候选项与其支持度,然后通过剪枝生成频繁项集},步骤主要为:

\begin{enumerate}

\item 找出所有的频繁项集(支持度大于等于给定的阈值);

\item 由频繁项集产生强关联规则(经过上个步骤后满足给定的置信度阈值的规则)。

\end{enumerate}

为验证Apriori算法实现的正确性,我先使用蛮力算法在模拟交易数据集上运行一次,将结果与Apriori算法的结果进行比较。验证算法的正确性后,在Groceries数据集上我则直接使用Apriori算法进行分析。


\section{实验及结果}

\subsection{模拟数据集}

我将支持度和置信度的阈值分别设置为0.1及0.6,蛮力算法的运行结果根据支持度和置信度排序后为:

\begin{lstlisting}[language=Python,
	keywordstyle=\color{blue!70},
	frame=shadowbox,
	breaklines=True]
------频繁项集------
('scissors', 'stapler') , 0.15
('plate', 'stapler') , 0.15
('knife', 'stapler') , 0.15
('scissors', 'knife') , 0.15
('stapler', 'glue') , 0.16
('ele_oven', 'TV') , 0.17
('milk',) , 0.18
('apple',) , 0.18
('bread',) , 0.19
('orange',) , 0.21
('beer',) , 0.22
('PC',) , 0.22
('phone',) , 0.22
('glue',) , 0.23
('plate',) , 0.23
('scissors',) , 0.25
('fridge',) , 0.25
('ele_oven',) , 0.26
('knife',) , 0.27
('TV',) , 0.29
('stapler',) , 0.30

------关联规则------
('scissors',) --> ('stapler',) , 0.60
('scissors',) --> ('knife',) , 0.60
('plate',) --> ('stapler',) , 0.65
('ele_oven',) --> ('TV',) , 0.65
('glue',) --> ('stapler',) , 0.70
\end{lstlisting}

得到以上的参考结果后,我使用Apriori算法和同样的参数对模拟交易数据集进行分析,所得结果见表 \ref{tab:apriori_sim_sup} 和 表 \ref{tab:apriori_sim_con}。

\begin{table}[H]
	\renewcommand\arraystretch{1.35}
	\caption{Apriori算法在模拟交易数据集下发现的频繁项集(按支持度排序)}
	\label{tab:apriori_sim_sup}
	\centering
	
	\begin{tabular}{c|c|c|c}
		\centering
		频繁项集 & 支持度 & 频繁项集 & 支持度 \\
		\hline
		stapler, scissors & 0.15 & PC & 0.22 \\
		knife, scissors & 0.15 & phone & 0.22 \\
		stapler, knife & 0.15 & glue & 0.23 \\
		plate, stapler & 0.15 &plate & 0.23 \\
		stapler, glue & 0.16 & scissors & 0.25 \\
		ele\_oven, TV & 0.17 & fridge & 0.25 \\
		milk & 0.18 & ele\_oven & 0.26 \\
		apple & 0.18 & knife & 0.27 \\
		bread & 0.19 & TV & 0.29 \\
		orange & 0.21 & stapler & 0.30 \\
		beer & 0.22 & & \\		
	\end{tabular}
\end{table}

\begin{table}[H]
	\renewcommand\arraystretch{1.35}
	\caption{Apriori算法在模拟交易数据集下发现的关联规则(按置信度排序)}
	\label{tab:apriori_sim_con}
	\centering
	
	\begin{tabular}{c|c}
		\centering
		关联规则 & 置信度 \\
		\hline
		scissors --> stapler & 0.60 \\
		scissors --> knife & 0.60 \\
		plate --> stapler & 0.65 \\
		ele\_oven --> TV & 0.65 \\
		glue --> stapler & 0.70 \\
	\end{tabular}
\end{table}

对比蛮力算法和Apriori算法排序后的结果,容易发现二者一致,可以证明我实现的Apriori算法的正确性。

\subsection{Groceries数据集}

确认Apriori算法实现的正确性后,我将其应用到真实数据上。考虑到Groceries数据集商品种类相对于实例较少的特点,我选择了较小的支持度(0.01),置信度选择为0.2。运行结果参见表 \ref{tab:apriori_gro_sup} 和表 \ref{tab:apriori_gro_con}。

\begin{table}[H]
	\renewcommand\arraystretch{1.35}
	\caption{Apriori算法在Groceries数据集下发现的频繁项集(按支持度排序)}
	\label{tab:apriori_gro_sup}
	\centering
	
	\begin{tabular}{c|c|c|c}
		\centering
		频繁项集 & 支持度 & 频繁项集 & 支持度 \\
		\hline
		stapler, scissors & 0.15 & PC & 0.22 \\
		knife, scissors & 0.15 & phone & 0.22 \\
		stapler, knife & 0.15 & glue & 0.23 \\
		plate, stapler & 0.15 &plate & 0.23 \\
		stapler, glue & 0.16 & scissors & 0.25 \\
		ele\_oven, TV & 0.17 & fridge & 0.25 \\
		milk & 0.18 & ele\_oven & 0.26 \\
		apple & 0.18 & knife & 0.27 \\
		bread & 0.19 & TV & 0.29 \\
		orange & 0.21 & stapler & 0.30 \\
		beer & 0.22 & & \\		
	\end{tabular}
\end{table}

\begin{table}[H]
	\renewcommand\arraystretch{1.35}
	\caption{Apriori算法在Groceries数据集下发现的关联规则(按置信度排序)}
	\label{tab:apriori_gro_con}
	\centering
	
	\begin{tabular}{c|c}
		\centering
		关联规则 & 置信度 \\
		\hline
		scissors --> stapler & 0.60 \\
		scissors --> knife & 0.60 \\
		plate --> stapler & 0.65 \\
		ele\_oven --> TV & 0.65 \\
		glue --> stapler & 0.70 \\
	\end{tabular}
\end{table}

为进一步利用Apriori算法探索Groceries数据集的关联规则,我尝试了不同的支持度和置信度,具体结果可参见图 \ref{fig:apriori}。

\section{结论}


\newpage
\begin{appendix}
	\section{附录}
	\subsection{模拟交易数据集的详细信息}
	\label{apd:dataset}
	
	我将模拟交易数据集的交易记录按照交易商品数分为4类,即2、3、4、5件。不同的商品件数按照不同的比例随机混合三类商品,具体混合规则可参见表 \ref{tab:dataset}。
	
	\begin{table}[H]
		\renewcommand\arraystretch{1.35}
		\caption{模拟交易数据集生成交易记录的混合规则}
		\label{tab:dataset}
		\centering
		
		\begin{tabular}{c|c}
			\centering
			商品数 & 混合规则(括号中数字代表各类商品在记录中所占数量) \\
			\hline
			2 & (2); (1, 1) \\
			3 & (3); (2, 1) \\
			4 & (4); (3, 1); (2, 1, 1) \\
			5 & (5); (4, 1); (3, 2); (2, 2, 1) \\		
		\end{tabular}
	\end{table}
	
	
	\subsection{Apriori算法实现代码}
	\label{apd:code}
	
	\begin{lstlisting}[language=Python,
	numbers=left,
	keywordstyle=\color{blue!70},
	frame=shadowbox,
	breaklines=True]
from sklearn.cross_validation import StratifiedKFold
from sklearn.datasets import make_classification
from sklearn.cross_validation import train_test_split
from sklearn.metrics import classification_report

from sklearn.tree import export_graphviz
from sklearn.tree import DecisionTreeClassifier
from sklearn.ensemble import BaggingClassifier
from sklearn.ensemble import AdaBoostClassifier
from sklearn.grid_search import GridSearchCV

import numpy as np
import matplotlib.pyplot as plt
plt.rcParams['font.sans-serif']=['SimHei']
plt.rcParams['axes.unicode_minus']=False

##################### making dataset ######################

n = 500                 # number of instances
n_f = 30                # number of features
n_c = 3                 # number of classes
inf_f = int(0.6 * n_f)  # 60% real features
red_f = int(0.1 * n_f)  # 10% redundant features
rep_f = int(0.1 * n_f)  # 10% repeated features
random_seed = 1         # random seed for the experiments

X, Y = make_classification(n_samples=n, n_classes=n_c, flip_y=0.03,
                    n_features=n_f, n_informative=inf_f, n_redundant=red_f,
                    n_repeated=rep_f, random_state=random_seed)
X_train, X_test, Y_train, Y_test = \
    train_test_split(X, Y, test_size=0.8, random_state=random_seed)

##################### making dataset ######################

# directly train the model with training set & testing set
def exp_plain_train():
    model = DecisionTreeClassifier(random_state=random_seed)
    model.fit(X_train, Y_train)

    pred = model.predict(X_test)
    print(classification_report(Y_test, pred))

    score = model.score(X_test, Y_test)
    print('plain train score in testing set: ', score)
    score = model.score(X_train, Y_train)
    print('plain train score in training set: ', score)

# test model with cross-validation
def exp_cv(folds=10):
    kfolds = StratifiedKFold(Y, n_folds=folds, random_state=random_seed)
    model = DecisionTreeClassifier(random_state=random_seed)
    scores_train = []
    scores_test = []
    for train, test in kfolds:
        model.fit(X[train], Y[train])
        # pred = model.predict(X[test])

        score = model.score(X[test], Y[test])
        scores_test.append(score)
        score = model.score(X[train], Y[train])
        scores_train.append(score)

    mean_test = np.array(scores_test).mean()
    mean_train = np.array(scores_train).mean()
    print('avg score with cv folds %d in testing set: '
          % folds, mean_test)
    print('avg score with cv folds %d in training set: '
          % folds, mean_train)

    return mean_test, mean_train

def plot_cv():
    mean_tests = []
    mean_trains = []
    folds_sum = 11
    for i in range(2, folds_sum):
        m1, m2 = exp_cv(i)
        mean_tests.append(m1)
        mean_trains.append(m2)

    # start to plot
    x = np.arange(2, folds_sum)
    total_width, n = 0.8, 2
    width = total_width / n
    x = x - (total_width - width) / 2

    plt.bar(x, mean_tests, width=width,
            facecolor='#9999ff', edgecolor='white', label=u'测试集')
    plt.bar(x + width, mean_trains, width=width,
            facecolor='#ffa07a', edgecolor='white', label=u'训练集')
    for x, y1, y2 in zip(x, mean_tests, mean_trains):
        plt.text(x - 0.05, y1 + 0.01, '%.2f' % y1, ha='center', va='bottom')
        plt.text(x+width - 0.05, y2 + 0.01, '%.2f' % y2, ha='center', va='bottom')

    plt.xlabel(u'折数')
    plt.ylabel(u'平均 f1-score')
    plt.ylim((0, 1.3))
    plt.legend()
    plt.savefig('report/img/cv_bar')
    plt.show()

# grid search
def exp_grid_search(folds=10):
    model = DecisionTreeClassifier(random_state=random_seed)
    param_grid = {'criterion': ['gini', 'entropy'],
                  'max_features': ['sqrt', 'log2', None],
                  'max_depth': list(range(3, 15)),
                  'presort': [True, False],
                  'splitter': ['best', 'random']
                  }
    grid = GridSearchCV(model, param_grid, cv=folds, scoring='f1_weighted')
    grid.fit(X, Y)

    print(grid.best_params_)
    print(grid.best_score_)

    export_graphviz(grid.best_estimator_, filled=True, out_file='report/img/gs.dot')

# bagging alg
def bagging(cv=True):
    bagging = BaggingClassifier(
        DecisionTreeClassifier(random_state=random_seed),
        n_estimators=5,            # number of models
        random_state=random_seed,
        bootstrap=True,
        max_samples=1.0,            # Bootstrap sample size radio
        bootstrap_features=True,
        max_features=1.0,           # Bootstrap feature usage radio
    )
    if cv:          # using cross-validation
        scores_train = []
        scores_test = []
        kfolds = StratifiedKFold(Y, n_folds=10, random_state=random_seed)
        for train, test in kfolds:
            bagging.fit(X[train], Y[train])

            score = bagging.score(X[test], Y[test])
            scores_test.append(score)
            score = bagging.score(X[train], Y[train])
            scores_train.append(score)

        mean_test = np.array(scores_test).mean()
        mean_train = np.array(scores_train).mean()
        print('avg score with cv folds 10 in testing set: ', mean_test)
        print('avg score with cv folds 10 in training set: ', mean_train)
    else:           # without cross-validation
        bagging.fit(X_train, Y_train)
        pred = bagging.predict(X_test)
        print(classification_report(Y_test, pred))

    # check the features extracted by each model
    plt.figure(figsize=(7, 5))
    f_n = 30
    x = list(range(1, f_n + 1))
    for i, f in enumerate(bagging.estimators_features_):
        print('model %d' % (i + 1), f)
        plt.scatter(x, f, label=u'子模型 %d' % (i + 1))
    plt.xlabel(u'特征编号')
    plt.xticks(list(range(0, 41, 5)))
    plt.ylabel(u'特征数值')
    plt.legend(loc=1)
    plt.savefig('report/img/bagging_feature_%d' % len(bagging.estimators_features_))
    plt.show()

def plot_bagging():
    # results
    x = list(range(10, 101, 10))
    y = [0.694, 0.720, 0.726, 0.730, 0.730, 0.738, 0.738, 0.738, 0.738, 0.744]
    y_b1 = [0.626, 0.672, 0.702, 0.716, 0.704, 0.727, 0.735, 0.732, 0.736, 0.738]
    y_b2 = [0.668, 0.711, 0.738, 0.734, 0.744, 0.746, 0.738, 0.730, 0.740, 0.738]

    # ploting code
    plt.figure(figsize=(6, 4))
    ax = plt.gca()
    ax.plot(x, y_b1, color='#90EE90', linewidth=1.7, label=u'70%特征')
    ax.plot(x, y_b2, color='#ffa07a', linewidth=1.7, label=u'90%特征')
    ax.plot(x, y, color='#9999ff', linewidth=1.7, label=u'100%特征')
    ax.scatter(x, y, s=13, c='#9999ff')
    ax.scatter(x, y_b1, s=13, c='#90EE90')
    ax.scatter(x, y_b2, s=13, c='#ffa07a')
    ax.grid(color='b', alpha=0.5, linestyle='dashed', linewidth=0.5)
    plt.xlim((5, 105))
    plt.xticks(x)
    plt.xlabel(u'子模型数量')
    plt.ylabel(u'平均 f1-score')
    plt.legend()
    plt.savefig('report/img/bagging_kline')
    plt.show()

# boosting alg
def boosting(cv=True):
    boosting = AdaBoostClassifier(
        DecisionTreeClassifier(max_depth=3, min_samples_leaf=2, random_state=random_seed),
        n_estimators=15,   # number of models
        algorithm='SAMME',  # Advanced-Boosting
        random_state=random_seed
    )
    if cv:      # using cross-validation
        scores_train = []
        scores_test = []
        kfolds = StratifiedKFold(Y, n_folds=10, random_state=random_seed)
        for train, test in kfolds:
            boosting.fit(X[train], Y[train])

            score = boosting.score(X[test], Y[test])
            scores_test.append(score)
            score = boosting.score(X[train], Y[train])
            scores_train.append(score)

        mean_test = np.array(scores_test).mean()
        mean_train = np.array(scores_train).mean()
        print('avg score with cv folds 10 in testing set: ', mean_test)
        print('avg score with cv folds 10 in training set: ', mean_train)
    else:       # without cross-validation
        boosting.fit(X_train, Y_train)

        pred = boosting.predict(X_train)
        print(classification_report(Y_train, pred))

        pred = boosting.predict(X_test)
        print(classification_report(Y_test, pred))

    # plot the relation between weights and error
    plt.figure()
    plt.xlabel(u'子模型权重')
    plt.ylabel(u'错误率')
    plt.plot(boosting.estimator_weights_, boosting.estimator_errors_)
    plt.savefig('report/img/boosting-weight-error-%d' % len(boosting.estimator_weights_))
    plt.show()

def plot_boosting():
    # results
    x = list(range(10, 101, 10))
    y1 = [0.621, 0.649, 0.649, 0.666, 0.666, 0.672, 0.672, 0.688, 0.688, 0.710]
    y2 = [0.608, 0.570, 0.596, 0.616, 0.635, 0.633, 0.629, 0.650, 0.664, 0.668]
    y1_tr = [0.860, 0.958, 0.984, 0.995, 0.996, 0.999, 1.000, 1.000, 1.000, 1.000]
    y2_tr = [0.904, 0.936, 0.960, 0.975, 0.988, 0.993, 0.993, 0.995, 0.997, 0.998]

    # plot testing results
    plt.figure(figsize=(6, 4))
    ax = plt.gca()
    ax.plot(x, y1, color='#9999ff', linewidth=1.7, label='SAMME.R')
    ax.plot(x, y2, color='#90EE90', linewidth=1.7, label='SAMME')
    ax.scatter(x, y1, s=13, c='#9999ff')
    ax.scatter(x, y2, s=13, c='#90EE90')
    ax.grid(color='b', alpha=0.5, linestyle='dashed', linewidth=0.5)
    plt.xlim((5, 105))
    plt.xticks(x)
    plt.xlabel(u'子模型数量')
    plt.ylabel(u'平均 f1-score')
    plt.legend()
    plt.savefig('report/img/boosting_kline_test')
    plt.show()

    # plot training results
    plt.figure(figsize=(6, 4))
    ax = plt.gca()
    ax.plot(x, y1_tr, color='#9999ff', linewidth=1.7, label='SAMME.R')
    ax.plot(x, y2_tr, color='#90EE90', linewidth=1.7, label='SAMME')
    ax.scatter(x, y1_tr, s=13, c='#9999ff')
    ax.scatter(x, y2_tr, s=13, c='#90EE90')
    ax.grid(color='b', alpha=0.5, linestyle='dashed', linewidth=0.5)
    plt.xlim((5, 105))
    plt.xticks(x)
    plt.xlabel(u'子模型数量')
    plt.ylabel(u'平均 f1-score')
    plt.legend()
    plt.savefig('report/img/boosting_kline_train')
    plt.show()

if __name__ == '__main__':
    # exp_plain_train()
    # exp_cv()
    # plot_cv()
    # exp_grid_search()
    boosting()
    # plot_boosting()
    # bagging()
    # plot_bagging()
	\end{lstlisting}
	
\end{appendix}

%========================================================================
\end{document}