\documentclass[12pt,a4paper]{article}
\usepackage{ctex}
\usepackage{amsmath,amscd,amsbsy,amssymb,latexsym,url,bm,amsthm}
\usepackage{epsfig,graphicx,subfigure}
\usepackage{enumitem,balance}
\usepackage{wrapfig}
\usepackage{mathrsfs, euscript}
\usepackage[usenames]{xcolor}
\usepackage{hyperref}
\usepackage[vlined,ruled,commentsnumbered,linesnumbered]{algorithm2e}
\usepackage{float}
\usepackage{array}
\usepackage{diagbox}
\usepackage{color}
\usepackage{indentfirst}
\usepackage{fancyhdr}
\usepackage{gensymb}
\usepackage{geometry}
\usepackage{setspace}
\usepackage{aurical}
\usepackage{times}
\usepackage{caption}
\usepackage{fontspec}
\usepackage{booktabs}
\usepackage{listings}
\usepackage{xcolor}
\setmainfont{Times New Roman}

\newtheorem{theorem}{Theorem}[section]
\newtheorem{lemma}[theorem]{Lemma}
\newtheorem{proposition}[theorem]{Proposition}
\newtheorem{corollary}[theorem]{Corollary}
\newtheorem{exercise}{Exercise}[section]
\newtheorem*{solution}{Solution}
\theoremstyle{definition}

\newcommand{\tabincell}[2]{\begin{tabular}{@{}#1@{}}#2\end{tabular}}
\newcommand{\postscript}[2]
 {\setlength{\epsfxsize}{#2\hsize}
  \centerline{\epsfbox{#1}}}

\renewcommand{\baselinestretch}{1.05}

\setlength{\oddsidemargin}{-0.365in}
\setlength{\evensidemargin}{-0.365in}
\setlength{\topmargin}{-0.3in}
\setlength{\headheight}{0in}
\setlength{\headsep}{0in}
\setlength{\textheight}{10.1in}
\setlength{\textwidth}{7in}
\makeatletter \renewenvironment{proof}[1][Proof] {\par\pushQED{\qed}\normalfont\topsep6\p@\@plus6\p@\relax\trivlist\item[\hskip\labelsep\bfseries#1\@addpunct{.}]\ignorespaces}{\popQED\endtrivlist\@endpefalse} \makeatother
\makeatletter
\renewenvironment{solution}[1][Solution] {\par\pushQED{\qed}\normalfont\topsep6\p@\@plus6\p@\relax\trivlist\item[\hskip\labelsep\bfseries#1\@addpunct{.}]\ignorespaces}{\popQED\endtrivlist\@endpefalse} \makeatother

\begin{document}

\section{引言}

近年来,人们逐渐从信息化时代迈向了数据时代,各种数据爆炸式地增长,数据消费也在日益增多,大量的信息、知识和利润隐藏在这些数据中。如何更有效地利用这些数据,已经成为这个时代下人们共同探索的问题之一。

\vspace{0.01\linewidth}
在这次大作业中,我将对Adult数据集进行全面的分析:首先探索数据集中各特征的分布信息;再划分数据集,尝试多种分类模型;最后比较这些模型在Adult数据集上的预测结果(分析代码均基于Python语言,相关工具和库包可参见附录 \ref{apd:tools})。

\section{探索Adult数据集}

\subsection{Adult数据集的基本信息}

%\vspace{0.01\linewidth}
Adult数据集 \cite{Dataset} 也称人口普查收入(Census Income)数据集,来源于美国1994年的人口普查数据库,可以作为二分类数据集,用来预测居民年收入是否超过50K\$,其基本信息可参见表 \ref{tab:basic-info}。

%\vspace{0.01\linewidth}
\begin{table}[H]
	\renewcommand\arraystretch{1.35}
	\caption{Adult数据集的基本信息}
	\label{tab:basic-info}
	\centering
	
	\begin{tabular}{c|c||c|c}
		\centering
		属性 & 值 & 属性 & 值 \\
		\hline
		\hline
		数据集特征 & 多变量 & 相关应用 & 分类 \\
		实例数 & 48842 & 捐赠日期 & 1996.5.1 \\
		领域 & 社会 & 是否有缺失值 & 有 \\
		属性特征 & 类别型或整数 & 官网访问次数 & 1188850 \\
		属性数目 & 14 & & \\
		
	\end{tabular}
\end{table}

\vspace{-0.01\linewidth}
Adult数据集的每个实例包含14个属性,其含义、数据类型、取值范围等基本信息见表 \ref{tab:feature-info}。

\begin{table}[H]
	\renewcommand\arraystretch{1.15}
	\caption{Adult数据集的基本信息}
	\label{tab:feature-info}
	\centering
	
	\begin{tabular}{c|c|c|c}
		\centering
		 特征名 & 含义 & 数据类型 & 类别数 \\
		\hline
		\hline
		age & 年龄 & 整数 & - \\
		workclass & 工作类型 & 类别型 & 8 \\
		fnlwgt & 序号 & 整数 & - \\
		education & 教育程度 & 类别型 & 16 \\
		education-num & 受教育时间 & 整数 & - \\
		marital-status & 婚姻状况 & 类别型 & 7 \\
		occupation & 职业 & 类别型 & 14 \\
		relationship & 家庭关系 & 类别型 & 6 \\
		race & 种族 & 类别型 & 5 \\
		sex & 性别 & 类别型 & 2 \\
		capital-gain & 资本收益 & 整数 & - \\
		capital-loss & 	资本损失 & 整数 & - \\
		hours-per-week & 每周工作小时数 & 整数 & - \\
		native-country & 原籍 & 类别型 & 41 \\
		
	\end{tabular}
\end{table}

\subsection{数据预处理}

\vspace{0.01\linewidth}
我首先使用pandas库读取Adult数据集,将其存储为pandas库中的DataFrame格式,随机打印出其中几个实例,对该数据集进行初步的观察,结果如下。

\vspace{0.015\linewidth}
	\begin{lstlisting}[
	numbers=left,
	keywordstyle=\color{blue!70},
	frame=shadowbox,
	breaklines=True]
 age work_class  fnlwgt   education  education_num     marital_status
  24    Private  269799   Assoc-voc             11       Never-married   
  35          ?  169809   Bachelors             13  Married-civ-spouse   
  51    Private  257126        10th              6  Married-civ-spouse   
  72    Private  107814     Masters             14       Never-married   
  33    Private  205950     HS-grad              9       Never-married   

      occupation    relationship    race    sex  capital_gain
 Exec-managerial   Not-in-family   White   Male             0   
               ?         Husband   White   Male             0   
    Craft-repair         Husband   White   Male             0   
  Prof-specialty   Not-in-family   White   Male          2329   
   Other-service       Own-child   White   Male             0   

 capital_loss  hours_per_week  native_country   income  
            0              40   United-States   <=50K.  
            0              20   United-States    <=50K
            0              40   United-States   <=50K.  
            0              60   United-States    <=50K  
            0              40   United-States    <=50K   
	\end{lstlisting}
	
\vspace{0.02\linewidth}
从以上的初步观察可以得知,Adult数据集存在数据缺失的情况(如第3行和10行的“?”),我对整个数据集进行统计后,发现数据集中共有3620个实例存在缺失值,而其中2799个实例的缺失值多于1个(表 \ref{tab:nan})。同时,我发现分类目标(income)的部分值存在歧义,“<=50K.”与“<=50K”属于同类,却被赋上不同标签,在后续离散化过程中(\ref{sec:discrete} 小节)我会进行处理。

\begin{table}[H]
	\renewcommand\arraystretch{1.35}
	\caption{Adult数据集缺失值分布}
	\label{tab:nan}
	\centering
	
	\begin{tabular}{c|c|c|c|c}
		\centering
		 & 无缺失值 & 缺失1个特征 & 缺失2个特征 & 缺失3个特征 \\
		\hline

		实例数 & 45222 & 821 & 2753 & 46 \\

	\end{tabular}
\end{table}

\vspace{0.01\linewidth}
考虑到数据集中存在缺失值的实例数较少(仅占总数的7.41\%),且缺失的均为类别型变量,若用一般的方式填补会带来较大的偏差,因此我选择将这些实例直接删除,清理缺失值后的Adult数据集包含45222个实例,且各类别型特征的类数并未因此受到影响(如native-country特征原本包含41类,处理后依然包含41类)。

\subsection{Adult数据集中各特征的分布}

对Adult数据集进行检查和清理后,我开始探索Adult数据集中各特征的分布,我将数值型特征和类别型特征分别处理:对于数值型特征,我主要关注其数字特征(如均值,方差等)及分布密度;对于类别型特征,我主要关注其具体的分布情况。

\subsubsection{数值型特征的分布}

Adult数据集中的数值型特征为:age, fnlwgt, education-num, capital-gain, capital-loss以及hours-per-week。我首先统计其均值、标准差等数字特征,相应结果如表 \ref{tab:num_feature_avg}所示。

\begin{table}[H]
	\renewcommand\arraystretch{1.5}
	\caption{Adult数据集数值型特征的数字特征}
	\label{tab:num_feature_avg}
	\centering
	
	\begin{tabular}{c|c|c|c|c|c|c}
		\centering
		 特征名 &  均值 & 标准差 & 最大值 & 最小值 &  上四分位数 & 下四分位数 \\
		\hline
		\hline
		age & 38.548 & 13.218 & 90.000 & 17.000 & 47.000 & 28.000 \\
		fnlwgt & 18976.470 & 10563.920 & 1490400.000 & 13492.000 & 237926.000 & 117388.200 \\
		education-num & 10.118 & 2.553 & 16.000 & 1.000 & 13.000 & 9.000 \\
		capital-gain & 1101.430 & 7506.430 & 99999.000 & 0.000 & 0.000 & 0.000 \\
		capital-loss & 88.595 & 404.956 & 4356.000 & 0.000 & 0.000 & 0.000 \\
		hours-per-week & 40.938 & 12.008 & 99.000 & 1.000 & 45.000 & 40.000 \\

	\end{tabular}
\end{table}

为更加直观地探索各数值型特征的分布趋势,我作出了相应的概率核密度分布图(高斯核),见图 \ref{fig:num_feature_dis}。

\begin{figure}[H]
	\centering
	\subfigure[age]{
		\includegraphics[width=0.31\linewidth]{img/age_dis.png}
	}
	\subfigure[fnlwgt]{
		\includegraphics[width=0.31\linewidth]{img/fnlwgt_dis.png}
	}
	\subfigure[education-num]{
		\includegraphics[width=0.31\linewidth]{img/edu_num_dis.png}
	}
	\subfigure[capital-gain]{
		\includegraphics[width=0.31\linewidth]{img/cap_in_dis.png}
	}
	\subfigure[capital-loss]{
		\includegraphics[width=0.31\linewidth]{img/cap_out_dis.png}
	}
	\subfigure[hours-per-week]{
		\includegraphics[width=0.31\linewidth]{img/hours_per_week_dis.png}
	}
	\caption{Adult数据集数值型特征概率核密度分布}
	\label{fig:num_feature_dis}
\end{figure}

\subsubsection{类别型特征的分布}

Adult数据集中的类别型特征包含:workclass, education, marital-status, occupation, relationship, race, sex以及native-country。我将其分布表示为条形图或饼图(图 \ref{fig:class_feature_dis})。各特征中包含的详细类名已记录于附录 \ref{apd:classes}中。

\begin{figure}[H]
	\centering
	\subfigure[education]{
		\includegraphics[width=0.43\linewidth]{img/edu_dis.png}
	}
	\subfigure[marital-status]{
		\includegraphics[width=0.43\linewidth]{img/marry_dis.png}
	}
	\subfigure[occupation]{
		\includegraphics[width=0.43\linewidth]{img/occupation_dis.png}
	}
	\subfigure[race]{
		\includegraphics[width=0.43\linewidth]{img/race_dis.png}
	}
	\subfigure[relationship]{
		\includegraphics[width=0.43\linewidth]{img/relationship_dis.png}
	}
	\subfigure[sex]{
		\includegraphics[width=0.43\linewidth]{img/sex_dis.png}
	}
	\subfigure[workclass]{
		\includegraphics[width=0.43\linewidth]{img/work_class_dis.png}
	}
	\subfigure[workclass]{
		\includegraphics[width=0.43\linewidth]{img/native_country_dis.png}
	}
	\caption{Adult数据集数值型特征概率核密度分布(由于native-country包含的类数较多,因此除United-States外国籍的分布见附录 \ref{apd:native-country})}
	\label{fig:class_feature_dis}
\end{figure}

\subsection{Adult数据集中各特征的相关性}

探索完Adult数据集中各特征的分布后,我开始探索特征之间的相关性,
\newpage
\begin{appendix}
	\section{附录}
	\subsection{作业中使用的工具及库包}
	\label{apd:tools}
	本次作业我所使用的编程语言为Python \cite{python},编辑环境以jupyter notebook \cite{notebook} 为主。作业中我使用的库包见表 \ref{tab:import}。
	
	\begin{table}[H]
		\renewcommand\arraystretch{1.35}
		\caption{本作业中使用的库包}
		\label{tab:import}
		\centering
		
		\begin{tabular}{c|c}
			\centering
			库包名 &  用途 \\
			\hline
	
			numpy \cite{numpy} & 处理数据,数值计算 \\
			pandas \cite{pandas} & 读取数据,绘图,处理数据 \\
			matplotlib \cite{matplotlib} & 绘图 \\
			scipy \cite{scipy} & 数值计算 \\
			scikit-learn \cite{sklearn} & 分类模型的构造和运算 \\
	
		\end{tabular}
	\end{table}
	
	\subsection{Adult数据集特征分布补充资料}
	
	\subsubsection{native-country特征的详细分布}
	\label{apd:native-country}
	
	\vspace{-0.02\linewidth}
	\begin{figure}[H]
		\centering
		\includegraphics[width=0.75\linewidth]{img/native_country_dis2.png}
		\caption{Adult数据集native-country特征分布(除United-States)}
		\label{fig:class_feature_dis2}
	\end{figure}
	
	\subsubsection{各类别型特征下的详细类名}
	\label{apd:classes}
	
	\begin{enumerate}
	
	\item \textbf{workclass} Private, Local-gov, Self-emp-not-inc, Federal-gov, State-gov, Self-emp-inc, Without-pay, Never-worked;
	
	\item \textbf{education} 11th, HS-grad, Assoc-acdm, Some-college, 10th, Prof-school, 7th-8th, Bachelors, Masters, Doctorate, 5th-6th, Assoc-voc, 9th, 12th, 1st-4th, Preschool;
	
	\item \textbf{marital-status} Never-married, Married-civ-spouse, Widowed, Divorced, Separated, Married-spouse-absent, Married-AF-spouse;
	
	\item \textbf{occupation} Machine-op-inspct, Farming-fishing, Protective-serv, Other-service, Prof-specialty, Craft-repair, Adm-clerical, Exec-managerial, Tech-support, Sales, Priv-house-serv, Transport-moving, Handlers-cleaners, Armed-Forces;
	
	\item \textbf{relationship} Own-child, Husband, Not-in-family, Unmarried, Wife, Other-relative;
	
	\item \textbf{race} Black, White, Asian-Pac-Islander, Other, Amer-Indian-Eskimo;
	
	\item \textbf{sex} Male, Female;
	
	\item \textbf{native-country} United-States, Cuba, Jamaica, India, Mexico, Puerto-Rico, Honduras, England, Canada, Germany, Iran, Philippines, Poland, Columbia, Cambodia, Thailand, Ecuador, Laos, Taiwan, Haiti, Portugal, Dominican-Republic, El-Salvador, France, Guatemala, Italy, China, South, Japan, Yugoslavia, Peru, Outlying-US(Guam-USVI-etc), Scotland, Trinadad\&Tobago, Greece, Nicaragua, Vietnam, Hong, Ireland, Hungary, Holand-Netherlands.
	       
	\end{enumerate}
	
\end{appendix}

\bibliographystyle{ieeetr}
\bibliography{bio}

%========================================================================
\end{document}