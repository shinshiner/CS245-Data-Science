\documentclass[12pt,a4paper]{article}
\usepackage{ctex}
\usepackage{amsmath,amscd,amsbsy,amssymb,latexsym,url,bm,amsthm}
\usepackage{epsfig,graphicx,subfigure}
\usepackage{enumitem,balance}
\usepackage{wrapfig}
\usepackage{mathrsfs, euscript}
\usepackage[usenames]{xcolor}
\usepackage{hyperref}
\usepackage[vlined,ruled,commentsnumbered,linesnumbered]{algorithm2e}
\usepackage{float}
\usepackage{array}
\usepackage{diagbox}
\usepackage{color}
\usepackage{indentfirst}
\usepackage{fancyhdr}
\usepackage{gensymb}
\usepackage{geometry}
\usepackage{setspace}
\usepackage{aurical}
\usepackage{times}
\usepackage{caption}
\usepackage{fontspec}
\usepackage{booktabs}
\usepackage{listings}
\usepackage{xcolor}
\setmainfont{Times New Roman}

\newtheorem{theorem}{Theorem}[section]
\newtheorem{lemma}[theorem]{Lemma}
\newtheorem{proposition}[theorem]{Proposition}
\newtheorem{corollary}[theorem]{Corollary}
\newtheorem{exercise}{Exercise}[section]
\newtheorem*{solution}{Solution}
\theoremstyle{definition}

\newcommand{\tabincell}[2]{\begin{tabular}{@{}#1@{}}#2\end{tabular}}
\newcommand{\postscript}[2]
 {\setlength{\epsfxsize}{#2\hsize}
  \centerline{\epsfbox{#1}}}

\renewcommand{\baselinestretch}{1.05}

\setlength{\oddsidemargin}{-0.365in}
\setlength{\evensidemargin}{-0.365in}
\setlength{\topmargin}{-0.3in}
\setlength{\headheight}{0in}
\setlength{\headsep}{0in}
\setlength{\textheight}{10.1in}
\setlength{\textwidth}{7in}
\makeatletter \renewenvironment{proof}[1][Proof] {\par\pushQED{\qed}\normalfont\topsep6\p@\@plus6\p@\relax\trivlist\item[\hskip\labelsep\bfseries#1\@addpunct{.}]\ignorespaces}{\popQED\endtrivlist\@endpefalse} \makeatother
\makeatletter
\renewenvironment{solution}[1][Solution] {\par\pushQED{\qed}\normalfont\topsep6\p@\@plus6\p@\relax\trivlist\item[\hskip\labelsep\bfseries#1\@addpunct{.}]\ignorespaces}{\popQED\endtrivlist\@endpefalse} \makeatother

\begin{document}

\section{引言}

近年来,人们逐渐从信息化时代迈向了数据时代,各种数据爆炸式地增长,数据消费也在日益增多,大量的信息、知识和利润隐藏在这些数据中。如何更有效地利用这些数据,已经成为这个时代下人们共同探索的问题之一。

\vspace{0.01\linewidth}
在这次大作业中,我将对Adult数据集进行全面的分析:首先探索数据集中各特征的分布信息;再划分数据集,尝试多种分类模型;最后比较这些模型在Adult数据集上的预测结果(分析代码均基于Python语言,相关工具和库包可参见附录 \ref{apd:tools})\footnote{本次大作业以及以往小作业的代码可参见我的github仓库:\href{https://github.com/shinshiner/CS245-Data-Science}{https://github.com/shinshiner/CS245-Data-Science}}。

\section{探索Adult数据集}

\subsection{Adult数据集的基本信息}

%\vspace{0.01\linewidth}
Adult数据集 \cite{Dataset} 也称人口普查收入(Census Income)数据集,来源于美国1994年的人口普查数据库,可以作为二分类数据集,用来预测居民年收入是否超过50K\$,其基本信息可参见表 \ref{tab:basic-info}。

%\vspace{0.01\linewidth}
\begin{table}[H]
	\renewcommand\arraystretch{1.35}
	\caption{Adult数据集的基本信息}
	\label{tab:basic-info}
	\centering
	
	\begin{tabular}{c|c||c|c}
		\centering
		属性 & 值 & 属性 & 值 \\
		\hline
		\hline
		数据集特征 & 多变量 & 相关应用 & 分类 \\
		实例数 & 48842 & 捐赠日期 & 1996.5.1 \\
		领域 & 社会 & 是否有缺失值 & 有 \\
		属性特征 & 类别型或整数 & 官网访问次数 & 1188850 \\
		属性数目 & 14 & & \\
		
	\end{tabular}
\end{table}

\vspace{-0.01\linewidth}
Adult数据集的每个实例包含14个属性,其含义、数据类型、取值范围等基本信息见表 \ref{tab:feature-info}。

\begin{table}[H]
	\renewcommand\arraystretch{1.15}
	\caption{Adult数据集的基本信息}
	\label{tab:feature-info}
	\centering
	
	\begin{tabular}{c|c|c|c}
		\centering
		 特征名 & 含义 & 数据类型 & 类别数 \\
		\hline
		\hline
		age & 年龄 & 整数 & - \\
		workclass & 工作类型 & 类别型 & 8 \\
		fnlwgt & 序号 & 整数 & - \\
		education & 教育程度 & 类别型 & 16 \\
		education-num & 受教育时间 & 整数 & - \\
		marital-status & 婚姻状况 & 类别型 & 7 \\
		occupation & 职业 & 类别型 & 14 \\
		relationship & 家庭关系 & 类别型 & 6 \\
		race & 种族 & 类别型 & 5 \\
		sex & 性别 & 类别型 & 2 \\
		capital-gain & 资本收益 & 整数 & - \\
		capital-loss & 	资本损失 & 整数 & - \\
		hours-per-week & 每周工作小时数 & 整数 & - \\
		native-country & 原籍 & 类别型 & 41 \\
		
	\end{tabular}
\end{table}

\subsection{数据预处理}

\vspace{0.01\linewidth}
我首先使用pandas库读取Adult数据集,将其存储为pandas库中的DataFrame格式,随机打印出其中几个实例,对该数据集进行初步的观察,结果如下。

\vspace{0.015\linewidth}
	\begin{lstlisting}[
	numbers=left,
	keywordstyle=\color{blue!70},
	frame=shadowbox,
	breaklines=True]
 age work_class  fnlwgt   education  education_num     marital_status
  24    Private  269799   Assoc-voc             11       Never-married   
  35          ?  169809   Bachelors             13  Married-civ-spouse   
  51    Private  257126        10th              6  Married-civ-spouse   
  72    Private  107814     Masters             14       Never-married   
  33    Private  205950     HS-grad              9       Never-married   

      occupation    relationship    race    sex  capital_gain
 Exec-managerial   Not-in-family   White   Male             0   
               ?         Husband   White   Male             0   
    Craft-repair         Husband   White   Male             0   
  Prof-specialty   Not-in-family   White   Male          2329   
   Other-service       Own-child   White   Male             0   

 capital_loss  hours_per_week  native_country   income  
            0              40   United-States   <=50K.  
            0              20   United-States    <=50K
            0              40   United-States   <=50K.  
            0              60   United-States    <=50K  
            0              40   United-States    <=50K   
	\end{lstlisting}
	
\vspace{0.02\linewidth}
从以上的初步观察可以得知,Adult数据集存在数据缺失的情况(如第3行和10行的“?”),我对整个数据集进行统计后,发现数据集中共有3620个实例存在缺失值,而其中2799个实例的缺失值多于1个(表 \ref{tab:nan})。同时,我发现分类目标(income)的部分值存在歧义,“<=50K.”与“<=50K”属于同类,却被赋上不同标签,在后续预处理过程中(\ref{sec:fix-target}小节)我会进行处理。

\begin{table}[H]
	\renewcommand\arraystretch{1.35}
	\caption{Adult数据集缺失值分布}
	\label{tab:nan}
	\centering
	
	\begin{tabular}{c|c|c|c|c}
		\centering
		 & 无缺失值 & 缺失1个特征 & 缺失2个特征 & 缺失3个特征 \\
		\hline

		实例数 & 45222 & 821 & 2753 & 46 \\

	\end{tabular}
\end{table}

\vspace{0.01\linewidth}
考虑到数据集中存在缺失值的实例数较少(仅占总数的7.41\%),且缺失的均为类别型变量,若用一般的方式填补会带来较大的偏差,因此我选择将这些实例直接删除,清理缺失值后的Adult数据集包含45222个实例,虽然由于删除数据导致workclass特征减少了一类(Never-worked),但相应的实例只有10个,可以忽略不计。

\subsection{Adult数据集中各特征的分布}

对Adult数据集进行检查和清理后,我开始探索Adult数据集中各特征的分布,我将数值型特征和类别型特征分别处理:对于数值型特征,我主要关注其数字特征(如均值,方差等)及分布密度;对于类别型特征,我主要关注其具体的分布情况。

\subsubsection{数值型特征的分布}
\label{sec:num_feature}

Adult数据集中的数值型特征为:age, fnlwgt, education-num, capital-gain, capital-loss以及hours-per-week。我首先统计其均值、标准差等数字特征,相应结果如表 \ref{tab:num_feature_avg}所示。

\begin{table}[H]
	\renewcommand\arraystretch{1.5}
	\caption{Adult数据集数值型特征的数字特征}
	\label{tab:num_feature_avg}
	\centering
	
	\begin{tabular}{c|c|c|c|c|c|c}
		\centering
		 特征名 &  均值 & 标准差 & 最大值 & 最小值 &  上四分位数 & 下四分位数 \\
		\hline
		\hline
		age & 38.548 & 13.218 & 90.000 & 17.000 & 47.000 & 28.000 \\
		fnlwgt & 18976.470 & 10563.920 & 1490400.000 & 13492.000 & 237926.000 & 117388.200 \\
		education-num & 10.118 & 2.553 & 16.000 & 1.000 & 13.000 & 9.000 \\
		capital-gain & 1101.430 & 7506.430 & 99999.000 & 0.000 & 0.000 & 0.000 \\
		capital-loss & 88.595 & 404.956 & 4356.000 & 0.000 & 0.000 & 0.000 \\
		hours-per-week & 40.938 & 12.008 & 99.000 & 1.000 & 45.000 & 40.000 \\

	\end{tabular}
\end{table}

上表的数据大致反映了各特征的分布情况,为更加直观地探索各数值型特征的分布趋势,我作出了相应的概率核密度分布图(高斯核),见图 \ref{fig:num_feature_dis}。

\begin{figure}[H]
	\centering
	\subfigure[age]{
		\includegraphics[width=0.31\linewidth]{img/age_dis.pdf}
	}
	\subfigure[fnlwgt]{
		\includegraphics[width=0.31\linewidth]{img/fnlwgt_dis.pdf}
	}
	\subfigure[education-num]{
		\includegraphics[width=0.31\linewidth]{img/edu_num_dis.pdf}
	}
	\subfigure[capital-gain]{
		\includegraphics[width=0.31\linewidth]{img/cap_in_dis.pdf}
	}
	\subfigure[capital-loss]{
		\includegraphics[width=0.31\linewidth]{img/cap_out_dis.pdf}
	}
	\subfigure[hours-per-week]{
		\includegraphics[width=0.31\linewidth]{img/hours_per_week_dis.pdf}
	}
	\caption{Adult数据集数值型特征概率核密度分布}
	\label{fig:num_feature_dis}
\end{figure}

\vspace{0.01\linewidth}
容易看出,Adult数据集的6个数值型特征接近于正态分布。通过进一步的观察,我发现capital-gain和capital-loss这两个特征的大部分取值均分布在0附近,仅通过概率核密度图无法了解两特征其余取值的分布情况。为更精确、详细地探索其分布,我做出了两特征对数值($log(x + 1)$)下相应的直方图(图 \ref{fig:cap-in-out})。

\begin{figure}[H]
	\centering
	\subfigure[capital-gain]{
		\includegraphics[width=0.41\linewidth]{img/cap-in-hist.pdf}
	}
	\subfigure[capital-loss]{
		\includegraphics[width=0.41\linewidth]{img/cap-out-hist.pdf}
	}
	\caption{capital-in和capital-out的分布直方图}
	\label{fig:cap-in-out}
\end{figure}

\subsubsection{类别型特征的分布}

Adult数据集中的类别型特征包含:workclass, education, marital-status, occupation, relationship, race, sex以及native-country。我将其分布表示为条形图或饼图(图 \ref{fig:class_feature_dis1}, \ref{fig:class_feature_dis2})。各特征中包含的详细类名已记录于附录 \ref{apd:classes}中。

\begin{figure}[H]
	\centering
	\subfigure[education]{
		\includegraphics[width=0.43\linewidth]{img/edu_dis.pdf}
	}
	\subfigure[marital-status]{
		\includegraphics[width=0.43\linewidth]{img/marry_dis.pdf}
	}
	\subfigure[occupation]{
		\includegraphics[width=0.43\linewidth]{img/occupation_dis.pdf}
	}
	\subfigure[race]{
		\includegraphics[width=0.43\linewidth]{img/race_dis.pdf}
		\label{fig:race}
	}
	\caption{Adult数据集类别型特征分布条形图}
	\label{fig:class_feature_dis1}
\end{figure}

\begin{figure}[H]
	\centering
	\subfigure[relationship]{
		\includegraphics[width=0.43\linewidth]{img/relationship_dis.pdf}
	}
	\subfigure[sex]{
		\includegraphics[width=0.43\linewidth]{img/sex_dis.pdf}
	}
	\subfigure[workclass]{
		\includegraphics[width=0.43\linewidth]{img/work_class_dis.pdf}
	}
	\subfigure[native-country]{
		\includegraphics[width=0.43\linewidth]{img/native_country_dis.pdf}
		\label{fig:native-country}
	}
	\caption{Adult数据集类别型特征分布饼图(由于native-country包含的类数较多,因此除United-States外国籍的分布见附录 \ref{apd:native-country})}
	\label{fig:class_feature_dis2}
\end{figure}

从图 \ref{fig:native-country}及图 \ref{fig:race}中可以看出,Adult数据集中的大部分成年人都来自于美国,且为白色人种,我认为这一点对于Adult数据集其余特征的分布有着重要的影响。

\subsubsection{所有属性的分布直方图}

直方图是统计学中最常用的统计报告图之一,能够对数据分布进行精确且直观的图形表示。

\vspace{0.01\linewidth}
对数据集特征分布的探索,一种图片能够反映的信息是有限的,如概率核密度图仅统计分布信息,无法表示具体的数值。为更加全面地呈现出Adult数据集中所有属性(包括特征和分类目标)的分布,我作出每个属性的直方图,并将其排列在一起。限于篇幅,我将其放在附录 \ref{apd:attri}中\footnote{更多特征分布图可参见附录 \ref{apd:dis_detail}}。

\subsection{Adult数据集中各特征的相关性}
\label{sec:cof}

探索完Adult数据集中各特征的分布后,我开始探索特征之间的相关性。我选择相关系数作为相关性的判断标准。为方便计算,我将类别型特征数字化,即用0至$n-1$(n表示该特征对应的类数)表示各类,14个特征相关性的直观表示可参见图 \ref{fig:heat}。

\begin{figure}[H]
	\centering
	\includegraphics[width=0.85\linewidth]{img/cof_heat.png}
	\caption{Adult数据集特征相关系数热度图}
	\label{fig:heat}
\end{figure}

从上图中容易发现,半数特征之间都有着一定的相关性,特别的是,education和education-num特征之间有着较强的相关性。从常识分析,有着相同教育程度的成年人也应具有相同的受教育时间,为验证这一猜测,我打印出100个实例的education和education-num,部分结果如表 \ref{tab:guess}。

\begin{table}[H]
	\renewcommand\arraystretch{1.35}
	\caption{部分实例的education和education-num}
	\label{tab:guess}
	\centering
	
	\begin{tabular}{c||c|c}
		\centering
		 & education & education-num \\
		\hline
		\hline
		
		0 & Some-college & 10 \\
		1 & Assoc-acdm & 12 \\
		2 & Bachelors & 13 \\
		3 & HS-grad & 9 \\
		4 & Bachelors & 13 \\
		5 & Assoc-voc & 11 \\
		6 & Masters & 14 \\
		7 & 11th & 7 \\
		8 & Bachelors & 13 \\
		9 & 9th & 5 \\

	\end{tabular}
\end{table}

根据实验结果,education与education-num特征确实一一对应,在进行分类时可以将其中之一删除,以防止冗余特征出现。

\vspace{0.01\linewidth}
除education与education-num外,性别-家庭关系、年龄-家庭关系、年龄-婚姻状况、家庭关系-每周工作小时数这几对特征均有着较强的相关性,而fnlwgt与其余特征之间都几乎无相关性。

\vspace{0.01\linewidth}
图 \ref{fig:heat}仅提供了特征间相关性的直观表达,其详细的数值可参见图 \ref{fig:heat2}。

\begin{figure}[H]
	\centering
	\includegraphics[width=0.85\linewidth]{img/cof_heat_anno.png}
	\caption{Adult数据集特征相关系数热度图(含相关系数值)}
	\label{fig:heat2}
\end{figure}

%\begin{table}[H]
%	\renewcommand\arraystretch{1.35}
%	\caption{Adult数据集特征相关系数(特征编号与图 \ref{fig:heat}对应)}
%	\label{tab:cof}
%	\centering
%	
%	\begin{tabular}{c|c|c|c|c|c|c|c|c|c|c|c|c|c|c}
%		\centering
%		
%		& 1 & 2 & 3 & 4 & 5 & 6 & 7 & 8 & 9 & 10 & 11 & 12 & 13 & 14 \\
%		\hline
%
%		1 & 0.01 & 0.01 & 0.01 & 0.01 & 0.01 & 0.01 & 0.01 & 0.01 & 0.01 & 0.01 & 0.01 & 0.01 & 0.01 & 0.01 \\
%		
%		2 & 0.01 & 0.01 & 0.01 & 0.01 & 0.01 & 0.01 & 0.01 & 0.01 & 0.01 & 0.01 & 0.01 & 0.01 & 0.01 & 0.01 \\
%		
%		3 & 0.01 & 0.01 & 0.01 & 0.01 & 0.01 & 0.01 & 0.01 & 0.01 & 0.01 & 0.01 & 0.01 & 0.01 & 0.01 & 0.01 \\
%		
%		4 & 0.01 & 0.01 & 0.01 & 0.01 & 0.01 & 0.01 & 0.01 & 0.01 & 0.01 & 0.01 & 0.01 & 0.01 & 0.01 & 0.01 \\
%		
%		5 & 0.01 & 0.01 & 0.01 & 0.01 & 0.01 & 0.01 & 0.01 & 0.01 & 0.01 & 0.01 & 0.01 & 0.01 & 0.01 & 0.01 \\
%		
%		6 & 0.01 & 0.01 & 0.01 & 0.01 & 0.01 & 0.01 & 0.01 & 0.01 & 0.01 & 0.01 & 0.01 & 0.01 & 0.01 & 0.01 \\
%		
%		7 & 0.01 & 0.01 & 0.01 & 0.01 & 0.01 & 0.01 & 0.01 & 0.01 & 0.01 & 0.01 & 0.01 & 0.01 & 0.01 & 0.01 \\
%		
%		8 & 0.01 & 0.01 & 0.01 & 0.01 & 0.01 & 0.01 & 0.01 & 0.01 & 0.01 & 0.01 & 0.01 & 0.01 & 0.01 & 0.01 \\
%		
%		9 & 0.01 & 0.01 & 0.01 & 0.01 & 0.01 & 0.01 & 0.01 & 0.01 & 0.01 & 0.01 & 0.01 & 0.01 & 0.01 & 0.01 \\
%		
%		10 & 0.01 & 0.01 & 0.01 & 0.01 & 0.01 & 0.01 & 0.01 & 0.01 & 0.01 & 0.01 & 0.01 & 0.01 & 0.01 & 0.01 \\
%		
%		11 & 0.01 & 0.01 & 0.01 & 0.01 & 0.01 & 0.01 & 0.01 & 0.01 & 0.01 & 0.01 & 0.01 & 0.01 & 0.01 & 0.01 \\
%		
%		12 & 0.01 & 0.01 & 0.01 & 0.01 & 0.01 & 0.01 & 0.01 & 0.01 & 0.01 & 0.01 & 0.01 & 0.01 & 0.01 & 0.01 \\
%		
%		13 & 0.01 & 0.01 & 0.01 & 0.01 & 0.01 & 0.01 & 0.01 & 0.01 & 0.01 & 0.01 & 0.01 & 0.01 & 0.01 & 0.01 \\
%		
%		14 & 0.01 & 0.01 & 0.01 & 0.01 & 0.01 & 0.01 & 0.01 & 0.01 & 0.01 & 0.01 & 0.01 & 0.01 & 0.01 & 0.01 \\
%
%	\end{tabular}
%\end{table}

\section{划分数据集并构造分类模型}

探索完Adult数据集中各特征的分布和相关性后,我开始对其进行训练集和测试集的划分并构造一系列分类模型。

\subsection{划分数据集}

对数据集的划分一般有两种方法:一是直接按照一定的比例将数据划分为训练集和测试集(需保证训练集和测试集中的类分布大致相同);二是使用分层交叉验证,将数据随机等分为$k$个不相交子集,执行$k$次训练与测试,根据$k$次迭代的平均表现评价模型的性能。

\vspace{0.01\linewidth}
本次作业中,为使评价结果更加精确,我主要使用分层交叉验证方法划分数据集,只在训练基线分类模型(baseline)时使用直接划分训练集和测试集的方法。

\subsection{构造分类模型}

在机器学习领域,用于分类的算法种类繁多,基本的分类算法包括了逻辑回归(Logistic Regression)、K近邻(KNN)、决策树、支持向量机(SVM)以及多层感知机(MLP)等。考虑到Adult数据集的特征维数并不高,且分类目标简单(二分类),我在本次作业中选择使用决策树、SVM以及MLP三种分类模型(图 \ref{fig:model-init})。除使用单独模型进行分类外,我尝试应用了模型集成的方法以提高相应的分类效果。

\begin{figure}[H]
	\centering
	\subfigure[决策树]{
		\includegraphics[width=0.26\linewidth]{img/dt.png}
	}
	\subfigure[SVM]{
		\includegraphics[width=0.28\linewidth]{img/svm.png}
	}
	\subfigure[MLP]{
		\includegraphics[width=0.28\linewidth]{img/mlp.png}
	}
	\caption{决策树、SVM和MLP模型的直观表示}
	\label{fig:model-init}
\end{figure}

\subsection{数据预处理}

在进行正式的分类之前,我对Adult数据集的特征和分类目标进行了一些预处理,以方便分类模型的训练。

\subsubsection{Z-score标准化(规范化)}

一般地,Z-score标准化有如下形式:
\begin{equation}
	y = \dfrac{x - \mu}{\sigma}
\end{equation}
其中$\mu$和$\sigma$分别代表原数据的均值和标准差。

\vspace{0.01\linewidth}
对于分类问题的数值型特征,经Z-score标准化后符合标准正态分布,即$N(0, 1)$,可以有效避免因数值过大导致的模型偏差,并能够加快模型的学习速率,这些优势在SVM和MLP等模型中表现得更加明显。

\vspace{0.01\linewidth}
除此之外,\ref{sec:num_feature}小节的结果表明,Adult数据集里的数值型特征大多近似服从正态分布,在这个条件下,Z-score标准化能够取得更良好的效果。若特征的分布与正态分布相差较大,则Z-score标准化反而会破坏原数据的分布,造成额外的偏差。

\subsubsection{向量化}

Adult数据集中存在8个类别型特征,且除教育程度外,这些特征的类别之间并无大小关系,如性别的男女之间不存在大小的区别。为方便模型的训练,我将教育程度(education)这个冗余的特征直接删除,将其余类别型特征从字符串转化为one-hot向量。经过向量化处理后的数据,每个实例包含88维特征。

\vspace{0.01\linewidth}
类似于 \ref{sec:cof} 小节,我同样作出各特征之间相关性的热度图,从图 \ref{fig:heat3}中可以清晰地分辨出特征的相关性:大部分相关性集中于图的左上角,而国籍特征较为独立(含具体相关系数的热度图见附录 \ref{apd:heat})。

\begin{figure}[H]
	\centering
	\includegraphics[width=0.85\linewidth]{img/cof_heat2.png}
	\caption{Adult数据集特征向量化后的相关系数热度图}
	\label{fig:heat3}
\end{figure}

\subsubsection{分类目标修正}
\label{sec:fix-target}

Adult数据集的分类目标为居民收入,分为两类:<=50K\$以及>50K\$。而数据集中有些实例的分类目标后多了“.”,如变为“<=50K.”,直接使用原数据训练将导致分类目标变为4类。因此,我将分类目标转化为-1和1,分别表示<=50K\$和>50K\$。另外值得注意的一点是Adult数据集包含34014个负样本(<=50K),而仅有11208个正样本。

\subsection{分类性能评价标准}

对于分类问题,评价分类性能的标准一般有3个:精确度(precision),召回率(recall)以及f1-score。其在二分类问题中的定义如下。

假设二分类问题的结果为:


\begin{table}[H]
	\centering
	\begin{tabular}{c|c|c}
		& 正类 &  负类 \\
		\hline
		\hline
	
		 分类正确 & TP & FP \\
		 分类错误 & FN & TN \\
	\end{tabular}
\end{table}

则精确度、召回率和f1-score分别定义为
\begin{equation}
	\begin{aligned}
	P = \dfrac{TP}{TP + FP} \\ \\
	R = \dfrac{TP}{TP + FN} \\ \\
	F1 = \dfrac{2PR}{P + R} \\ \\
	\end{aligned}
\end{equation}

\vspace{-0.03\linewidth}
一般地,在大规模的数据集下,精确度和召回率会相互制约,出现一高一低的现象,而f1-score可以兼顾二者,有效减少因一者过大带来的误差。因此,在本次作业的实验中,我主要以\textbf{f1-score}作为分类性能的主要评价标准,精确度和召回率则在训练基线模型时作为参考标准。

\section{各分类模型的预测结果比较}
\label{sec:model-single}

\subsection{现有模型效果调查}

在开始训练分类模型前,我先调查了一些传统分类模型在Adult数据集上的分类效果 \cite{bench},衡量的标准为分类精确度,结果可参见表 \ref{tab:bench}。

\begin{table}[H]
	\centering
	\begin{tabular}{c|c}
		模型(算法) & 错误率(1-精确度) \\
		\hline
		\hline
	
		C4.5 & 15.54 \% \\
		C4.5-auto & 14.46 \% \\
		C4.5 rules & 14.94 \% \\
		Voted ID3 (0.6) & 15.64 \% \\
		Voted ID3 (0.8) & 16.47 \% \\
		T2 & 16.84 \% \\
		1R & 19.54 \% \\
		NBTree & 14.10 \% \\
		CN2 & 16.00 \% \\
		HOODG & 14.82 \% \\
		FSS Naive Bayes & 14.05 \% \\
		IDTM (Decision table) & 14.46 \% \\
		Naive-Bayes & 16.12 \% \\
		Nearest-neighbor (1) & 21.42 \% \\
		Nearest-neighbor (3) & 20.35 \% \\
		OC1 & 15.04 \% \\
		Pebls & 100 \% \\
		
	\end{tabular}
	\caption{传统分类模型在Adult数据集上的分类效果}
	\label{tab:bench}
\end{table}

\subsection{基线模型(baseline)}

为方便之后的比较,我首先使用sklearn \cite{sklearn}中的默认参数,不使用标准化,构造了三个基线模型,以及一个空模型(按相等概率 随机预测),参见表 \ref{tab:baselines1}和表 \ref{tab:baselines2}。

\begin{table}[H]
	\renewcommand\arraystretch{1.35}
	\caption{基线模型在Adult数据集上的性能(0.8-0.2比例的训练集-测试集分割)}
	\label{tab:baselines1}
	\centering
	
	\begin{tabular}{c|c|c|c|c}
		\centering
		 & 精确度(precision) & 召回率(recall) & f1-score & 时间(秒) \\
		\hline
		\hline
		
		决策树 & 0.81 & 0.81 & 0.81 & 0.25 \\
		SVM & 0.76 & 0.83 & 0.76 & 1044.22 \\
		MLP & 0.78 & 0.80 & 0.79 & 0.59 \\
		空模型 & 0.72 & 0.52 & 0.57 & 0.07 \\

	\end{tabular}
\end{table}

\begin{table}[H]
	\renewcommand\arraystretch{1.35}
	\caption{基线模型在Adult数据集上的性能(5折分层交叉验证)}
	\label{tab:baselines2}
	\centering
	
	\begin{tabular}{c|c|c|c|c}
		\centering
		 & 精确度(precision) & 召回率(recall) & f1-score & 时间(秒) \\
		\hline
		\hline
		
		决策树 & 0.80 & 0.80 & 0.80 & 4.41 \\
		SVM & 0.83 & 0.83 & 0.83 & 4491.21 \\
		MLP & 0.81 & 0.83 & 0.80 & 11.82 \\
		空模型 & 0.72 & 0.50 & 0.56 & 0.19 \\

	\end{tabular}
\end{table}

\subsection{决策树模型}

在本节中,我将使用网格搜索(Grid Search)对决策树模型的分类效果进行评估(5折交叉验证),搜索的参数及范围参见表 \ref{tab:dt-gs-para}。

\begin{table}[H]
	\renewcommand\arraystretch{1.35}
	\caption{对决策树模型网格搜索的参数及范围}
	\label{tab:dt-gs-para}
	\centering
	
	\begin{tabular}{c|c|c}
		\centering
		参数 & 含义 & 类型(范围) \\
		\hline
		\hline
		
		criterion & 特征选择的度量标准 & gini, entropy \\
		\hline
		max\_depth & 树的最大深度 & 正整数 \\
		\hline
		max\_features & \tabincell{c}{寻求最佳划分时\\要考虑的特征数目} & \tabincell{c}{总特征数或其平方根\\或其以2为底的对数值} \\
		\hline
		presort & 是否对数据预先排序 & 布尔值 \\
		\hline
		splitter & 结点划分策略 & best, random \\

	\end{tabular}
\end{table}

经过网格搜索后,我得到的最佳决策树模型的参数为:\{criterion: entropy, splitter: best, max\_features: 总特征数, max\_depth: 9, presort: True\};该模型在测试集上的平均f1-score为0.83。

\vspace{0.01\linewidth}
对于经过Z-score标准化的Adult数据集,网格搜索的结果与未标准化时并没有差别。

\vspace{0.01\linewidth}
最终通过网格搜索筛选出的最佳决策树模型的混淆矩阵及模型本身部分可视化分别见图 \ref{fig:cm}, \ref{fig:dt-best-vis}。

\begin{figure}[H]
	\centering
	\includegraphics[width=0.75\linewidth]{img/dt_cm.png}
	\caption{最佳决策树模型的混淆矩阵}
	\label{fig:cm}
\end{figure}

\begin{figure}[H]
	\centering
	\subfigure[决策树左部]{
		\includegraphics[width=0.85\linewidth]{img/dt_left.png}
	}
	\subfigure[决策树右部]{
		\includegraphics[width=0.85\linewidth]{img/dt_right.png}
	}
	\caption{最佳决策树模型的部分可视化}
	\label{fig:dt-best-vis}
\end{figure}

\subsection{SVM}

从表 \ref{tab:baselines1} 和表 \ref{tab:baselines2}来看,SVM的二分类能力并没有完全展现出来,并且难以收敛。在这一节中,我将尝试改变SVM模型的多个关键参数,尽可能改善其表现(模型表现均基于5折交叉验证)。

\subsubsection{核函数}

原始的SVM属于线性分类器,核函数的引入将SVM的应用推广到了非线性数据上。不同的核函数所需要的计算量和性能均有较大差别,我在Adult数据集上尝试应用四种常用的核函数:rbf核函数,多项式核函数(poly kernel),sigmoid核函数以及线性核函数,各模型相应的表现见图 \ref{fig:svm-kernel1} 及图 \ref{fig:svm-kernel2}。

\begin{figure}[H]
	\centering
	\subfigure[训练集]{
		\includegraphics[width=0.41\linewidth]{img/svm_kernel_tr.png}
	}
	\subfigure[测试集]{
		\includegraphics[width=0.41\linewidth]{img/svm_kernel_t.png}
	}
	\subfigure[训练集(标准化)]{
		\includegraphics[width=0.41\linewidth]{img/svm_kernel_ntr.png}
	}
	\subfigure[测试集(标准化)]{
		\includegraphics[width=0.41\linewidth]{img/svm_kernel_nt.png}
	}
	\caption{不同核函数下SVM模型的分类表现}
	\label{fig:svm-kernel1}
\end{figure}

\vspace{0.01\linewidth}
从上图中容易看出,对于未经Z-score标准化后的数据,SVM模型在使用rbf核时分类效果较佳,而其余核函数的分类效果极差,甚至低于空模型;而对于Z-score标准化后的数据,linear和sigmoid核函数的表现略有提高,但训练均极不稳定,向空模型的方向收敛。因此之后的实验均基于未Z-score标准化后的数据使用rbf核进行。

\subsubsection{惩罚系数}
\label{sec:penalty}

SVM在面对轻微的线性不可分数据时,可以通过引入惩罚系数$C$,将原有的优化目标
\begin{equation}
	L = \dfrac{1}{2}w^Tw
\end{equation}
变为
\begin{equation}
	L = \dfrac{1}{2}w^Tw+C\sum\limits_{n=1}^{N}\xi_n,
\end{equation}

适当的惩罚系数可以极大地改善SVM的分类性能。我使用类似于网格搜索的方式,遍历了各数量级的$C$取值,相应的结果可参见图 \ref{fig:penalty}。

\begin{figure}[H]
	\centering
	\includegraphics[width=0.75\linewidth]{img/svm_penalty.png}
	\caption{不同惩罚系数下SVM模型的分类表现}
	\label{fig:penalty}
\end{figure}

上图结果表明,过低的惩罚系数$C$对SVM在Adult数据集上的分类性能没有改善作用,0.6-1.0之间的$C$能够略微改善SVM的性能,而过大的$C$会让SVM趋于过拟合,泛化能力逐渐降低。

\subsubsection{Gamma ($\gamma$)}

rbf核的数学表达形式为:

\begin{equation}
	K(x_i, x_j)=exp(-\gamma\|x_i-x_j\|^2), \gamma>0
\end{equation}

$\gamma$是rbf核函数的一个重要参数,在sklearn中,其默认值为:$\dfrac{1}{\mbox{特征总维数}}$,在Adult数据集上即为$1/104=0.096$。我利用类似网格搜索的方式探索了各数量级下的$\gamma$对应的SVM模型的分类表现(表 \ref{tab:gammma})。

\begin{table}[H]
	\renewcommand\arraystretch{1.35}
	\caption{不同数量级的$\gamma$对应的SVM模型的分类表现}
	\label{tab:gammma}
	\centering
	
	\begin{tabular}{c|c|c||c|c|c}
		\centering
		$\gamma$ & 训练集f1-score & 测试集f1-score & $\gamma$ & 训练集f1-score & 测试集f1-score \\
		\hline
		\hline
		
		1 $\times 10^{-6}$  & 0.863 & 0.836 & 0.003 & 0.930 & 0.828 \\
		3 $\times 10^{-6}$ & 0.869 & 0.837 & 0.006 & 0.947 & 0.828 \\
		6 $\times 10^{-6}$ & 0.874 & 0.837 & 0.01 & 0.959 & 0.829 \\
		1 $\times 10^{-5}$  & 0.876 & 0.836 & 0.03 & 0.977 & 0.831 \\
		3 $\times 10^{-5}$ & 0.880 & 0.835 & 0.06 & 0.988 & 0.830 \\
		6 $\times 10^{-5}$ & 0.883 & 0.833 & 0.1 & 0.993 & 0.831 \\
		1 $\times 10^{-4}$ & 0.885 & 0.832 & 0.3 & 0.999 & 0.832 \\
		3 $\times 10^{-4}$ & 0.890 & 0.831 & 0.6 & 1.000 & 0.831 \\
		6 $\times 10^{-4}$ & 0.898 & 0.829 & 1.0 & 1.000 & 0.831 \\
		0.001 & 0.907 & 0.828 & 3.0 & 1.000 & 0.831 \\

	\end{tabular}
\end{table}

上表结果说明,

\subsubsection{小结}

经过以上的探索,我得到了SVM模型在Adult数据集分类上表现较佳的一组参数:\{核函数:rbf,惩罚参数:1.0,$\gamma$:xxx,是否进行Z-score标准化:否\},相应的SVM模型在测试集上的f1-score为:xxx。

\subsection{MLP}

从MLP基线模型(表 \ref{tab:baselines1}, \ref{tab:baselines2})的性能分析,MLP的性能相比决策树模型稍逊。我认为较大的原因在于MLP模型的默认参数并不适合Adult数据集,因此,在本节中,我将对MLP中不同的组件(激活函数,优化算法等)对其在Adult数据集上分类性能的影响进行探究。

\subsubsection{激活函数}

激活函数作用于MLP隐藏层中的每个神经元上,为MLP引入非线性因素,若缺少激活函数,神经网络的表达能力将极其有限。因此,激活函数是整个MLP中重要的组件之一。常用的激活函数包括relu \cite{{relu}}, tanh以及sigmoid等,相应的预测表现见图 \ref{fig:mlp-activate1}。

\begin{figure}[H]
	\centering
	\subfigure[训练集]{
		\includegraphics[width=0.41\linewidth]{img/mlp_activation_tr.png}
	}
	\subfigure[测试集]{
		\includegraphics[width=0.41\linewidth]{img/mlp_activation_t.png}
	}
	\subfigure[训练集(标准化)]{
		\includegraphics[width=0.41\linewidth]{img/mlp_activation_ntr.png}
	}
	\subfigure[测试集(标准化)]{
		\includegraphics[width=0.41\linewidth]{img/mlp_activation_nt.png}
	}
	\caption{不同激活函数下MLP模型的分类表现}
	\label{fig:mlp-activate1}
\end{figure}

上图结果表明,相比于sigmoid和tanh,relu激活函数在Adult数据集下的分类表现更佳。同时,MLP在经过Z-score标准化后的Adult数据集上能取得更好的分类效果,不仅提高了整体的分类f1-score,更有效地避免了使用relu激活函数时进入局部最小值的情况。因此,之后MLP部分的实验均基于Z-score标准化后的Adult数据集进行。

\subsubsection{优化算法}

优化算法作用于MLP更新参数时,对于同样的梯度分布,不同的优化算法的优化路径存在着极大的差别,图 \ref{fig:optim-ref} 形象地说明了这一点。因此,探究不同优化算法下MLP的分类性能是很有必要的,详细结果可参见图 \ref{fig:optim1}。

\begin{figure}[H]
	\centering
	\includegraphics[width=0.5\linewidth]{img/optim_example.png}
	\caption{不同优化算法的优化效果对比 \cite{optim_example}}
	\label{fig:optim-ref}
\end{figure}

\begin{figure}[H]
	\centering
	\subfigure[训练集(标准化)]{
		\includegraphics[width=0.41\linewidth]{img/mlp_optimizer_ntr.png}
	}
	\subfigure[测试集(标准化)]{
		\includegraphics[width=0.41\linewidth]{img/mlp_optimizer_nt.png}
	}
	\caption{不同优化算法下MLP模型的分类表现}
	\label{fig:optim1}
\end{figure}

根据上图结果,adam \cite{adam}、lbfgs与sgd三种优化算法均能使MLP在Adult数据集上取得较佳的分类表现,但adam的效果明显优于lbfgs与sgd,为探究其中的原因,我查阅了相关资料和原论文,认为图中反映出adam算法的巨大优势有着其坚实的理论依据。

\vspace{0.01\linewidth}
作为最常用的优化算法,adam能够利用较少的计算资源有效处理高噪声或稀疏的梯度分布,除此之外,其在优化的过程中能够通过计算梯度的一阶矩估计和二阶矩估计,为不同的参数设计独立的自适应性学习率。

\subsubsection{学习率}

在MLP的参数更新过程中,学习率决定了参数更新的速率,过大的学习率会导致模型进入局部最小值,而学习率过小则会减缓模型的学习速度。在本节中,我采用类似于 \ref{sec:penalty}小节的方式,对基于adam优化算法的MLP模型在不同学习率下的分类效果进行探索(图 \ref{fig:lr}),由于adam算法能够自适应地改变学习率,此处的学习率即指\textbf{初始学习率}。

\begin{figure}[H]
	\centering
	\includegraphics[width=0.75\linewidth]{img/mlp_lr.png}
	\caption{不同初始学习率下MLP模型的分类表现}
	\label{fig:lr}
\end{figure}

容易看出,除了过大的学习率有降低MLP性能的趋势外,MLP在其余学习率下的表现并无太大差别,考虑到实际应用时的时间占用,我认为0.03是基于adam优化算法的MLP模型在Adult数据集分类问题下较佳的学习率。

\subsubsection{小结}

经过以上的探索,我得到了MLP模型在Adult数据集分类上表现较佳的一组参数:\{激活函数:relu,优化算法:adam,学习率:0.03,是否进行Z-score标准化:是\},相应的MLP模型在测试集上的f1-score为:xxx。

\subsection{模型集成}

从以上几节的结果分析,决策树模型在Adult数据集上的分类效果弱于SVM和MLP。一般地,从一系列模型$M_1, M_2, ... , M_k$创建组合模型$M^*$,可以有效提高原模型的效果。相比于SVM和MLP,决策树模型训练时速度快、占用计算空间和资源较少,适合进行模型集成。

\vspace{0.01\linewidth}
因此,在本节中,我使用bagging和boosting两种模型集成的方法尝试改善决策树模型的分类效果。结果参见图 \ref{fig:model-plus}。

\begin{figure}[H]
	\centering
	\subfigure[bagging]{
		\includegraphics[width=0.31\linewidth]{img/bagging_kline.png}
	}
	\subfigure[boosting(SAMME)]{
		\includegraphics[width=0.31\linewidth]{img/boosting_kline_samme.png}
	}
	\subfigure[boosting(SAMME.R)]{
		\includegraphics[width=0.31\linewidth]{img/boosting_kline_sammer.png}
	}
	\caption{集成后的决策树模型的分类表现}
	\label{fig:model-plus}
\end{figure}

相比于模型单独工作,有些集成决策树模型的分类效果确实有些许提高(基于SAMME.R的boosting集成出现了过拟合),但是考虑到集成模型时带来的额外计算和存储开销,我认为在Adult数据集上使用集成决策树模型的实际收益并不如模型单独工作。

\vspace{0.01\linewidth}
为更加直观地了解各子模型的属性,我将三种集成方式下子模型提取的特征(或权重与错误率关系)可视化(图 \ref{fig:model-plus2})。

\begin{figure}[H]
	\centering
	\subfigure[bagging]{
		\includegraphics[width=0.31\linewidth]{img/bagging_feature_10.png}
	}
	\subfigure[boosting(SAMME)]{
		\includegraphics[width=0.31\linewidth]{img/boosting-weight-error-10.png}
	}
	\subfigure[boosting(SAMME.R)]{
		\includegraphics[width=0.31\linewidth]{img/boosting-weight-error-10-r.png}
	}
	\caption{三种集成方式下子模型提取的特征(或权重与错误率关系)可视化}
	\label{fig:model-plus2}
\end{figure}

\subsection{三种分类模型效果对比}
\label{sec:compare}

通过第 \ref{sec:model-single}小节的实验,我对决策树、SVM和MLP三种模型在Adult数据集上的分类效果有了大致的了解。在本节中,我将探索三种分类模型在不同条件下对Adult数据集的分类效果。

\subsubsection{Z-score标准化}

在前文的探索过程中,我已分别得到了三种模型在Z-score标准化前后的数据上的分类表现,在本节中我将这些结果进行一个直观的表示,方便观察在Adult数据集下,Z-score标准化对于三种模型的适合程度,如表 \ref{tab:norm}。

\begin{table}[H]
	\renewcommand\arraystretch{1.35}
	\caption{三种分类模型在Z-score标准化前后的数据上的分类表现}
	\label{tab:norm}
	\centering
	
	\begin{tabular}{c|c|c}
		\centering
		 & 测试集f1-score(标准化前) & 测试集f1-score(标准化后) \\
		\hline
		\hline
		
		决策树 & 0.83 & 0.83 \\
		SVM & & \\
		MLP & & \\	

	\end{tabular}
\end{table}

\subsubsection{PCA}

从 \ref{sec:cof}小节的结果可知,Adult数据集的特征之间存在一定的相关性,除education外,可能仍存在冗余的特征,因此我在本小节中对\textbf{原数据}进行PCA降维,并探索三种分类模型在降维后数据上的分类表现(表 \ref{tab:pca1})。

\begin{table}[H]
	\renewcommand\arraystretch{1.35}
	\caption{决策树、SVM和MLP在降维后测试集上的分类表现}
	\label{tab:pca1}
	\centering
	
	\begin{tabular}{c|c|c|c}
		\centering
		 & 降至1维 & 降至2维 & 降至3维 \\
		\hline
		\hline
		
		协方差总和 & 0.995091 & 0.999985 & 0.999999 \\
		决策树 & 0.83 & 0.83 & \\
		SVM & & & \\
		MLP & & & \\

	\end{tabular}
\end{table}

由PCA降维以及 \ref{sec:cof} 小节的结果可知,Adult数据集的特征包含了许多冗余的信息,仅取1维的主成分也足够表达绝大部分原信息。因此,三种模型在降维后的Adult数据集上表现与降维前并无差别。

%\begin{figure}[H]
%	\centering
%	\subfigure[bagging]{
%		\includegraphics[width=0.31\linewidth]{img/mlp_optimizer_ntr.png}
%	}
%	\subfigure[boosting(SAMME)]{
%		\includegraphics[width=0.31\linewidth]{img/mlp_optimizer_nt.png}
%	}
%	\subfigure[boosting(SAMME.R)]{
%		\includegraphics[width=0.31\linewidth]{img/mlp_optimizer_nt.png}
%	}
%	\caption{决策树、SVM和MLP在降维后数据上的分类表现}
%	\label{fig:pca1}
%\end{figure}

\subsubsection{最佳性能对比}

根据 \ref{sec:model-single} 与 \ref{sec:compare}小节的实验结果,三种模型均存在一个最佳的f1-score值,一定程度上代表其在Adult数据集上分类能力的极限(由于模型的参数空间极其巨大,无法保证该值是否为全局最优,但在一定程度上能反映该模型的最佳性能),对比参见表 \ref{tab:final_compare}。

\begin{table}[H]
	\renewcommand\arraystretch{1.35}
	\caption{三种分类模型在Adult数据集下的最佳分类性能}
	\label{tab:final_compare}
	\centering
	
	\begin{tabular}{c|c|c}
		\centering
		 & 训练集f1-score & 测试集f1-score \\
		\hline
		\hline
		
		决策树 & 0.85 & 0.83 \\
		SVM & & \\
		MLP & & \\	

	\end{tabular}
\end{table}

%\begin{figure}[H]
%	\centering
%	\subfigure[决策树]{
%		\includegraphics[width=0.23\linewidth]{img/dt.png}
%	}
%	\subfigure[SVM]{
%		\includegraphics[width=0.23\linewidth]{img/svm.png}
%	}
%	\subfigure[MLP]{
%		\includegraphics[width=0.23\linewidth]{img/mlp.png}
%	}
%	\subfigure[空模型]{
%		\includegraphics[width=0.23\linewidth]{img/mlp.png}
%	}
%	\subfigure[决策树]{
%		\includegraphics[width=0.23\linewidth]{img/dt.png}
%	}
%	\subfigure[SVM]{
%		\includegraphics[width=0.23\linewidth]{img/svm.png}
%	}
%	\subfigure[MLP]{
%		\includegraphics[width=0.23\linewidth]{img/mlp.png}
%	}
%	\subfigure[空模型]{
%		\includegraphics[width=0.23\linewidth]{img/mlp.png}
%	}
%	\subfigure[决策树]{
%		\includegraphics[width=0.23\linewidth]{img/dt.png}
%	}
%	\subfigure[SVM]{
%		\includegraphics[width=0.23\linewidth]{img/svm.png}
%	}
%	\subfigure[MLP]{
%		\includegraphics[width=0.23\linewidth]{img/mlp.png}
%	}
%	\subfigure[空模型]{
%		\includegraphics[width=0.23\linewidth]{img/mlp.png}
%	}
%	\caption{基线模型(第一、二行的数据集划分分别为0.8-0.2,以及0.7-0.3的训练集-测试集划分,第三行则使用了5折分层交叉验证}
%	\label{fig:baselines}
%\end{figure}

\section{对Adult数据集的分析结论}

Adult数据集是一个中等规模,存在少部分缺失值的二分类数据集,共48842个实例(3620个包含缺失值),每个实例包含14个特征,其中8个为类别型特征,6个为数值型特征。各类别型特征的分布差异较大,数值型特征的分布近似于正态分布。

\vspace{0.01\linewidth}
对Adult数据集特征的相关性分析表明,半数特征之间都存在着一定的相关性,同时数据集中存在冗余的特征(education和education-num),也存在和其余特征几乎完全独立的特征(fnlwgt)。

\vspace{0.01\linewidth}
我使用了决策树、SVM和MLP三种分类模型对Adult数据集进行分类,并根据f1-score为分类效果衡量标准,对模型的参数进行微调。结果表明,xxx的分类效果最佳,最高的f1-score可达到xxx,而xxx模型所消耗的时间最少,平均不到xxx秒。除使用单独模型进行分类外,我还对决策树模型进行了模型集成(bagging和boosting),集成后的模型能够稍微提高分类性能。

\section{致谢}

一个学期转眼即过,这门课程也接近尾声。我认为这门课和我以前上的绝大多数课都不同,陆老师在课堂上不仅有理论上简明易懂的讲解,更用大量的实例向我们展示了如何将这些理论运用到实际中去,这正是以往许多课程没有提供的。经过这一学期对这门课程的学习,我不知不觉地在应用知识中提升了对理论知识的理解,同时也感到自己的工程能力有了提升。

\vspace{0.01\linewidth}
最后,我要感谢陆老师在课堂上的精妙讲解和实例示范,感谢陆老师和助教在这门课程上对我的帮助!

\newpage
\begin{appendix}
	\section{附录}
	\subsection{作业中使用的工具及库包}
	\label{apd:tools}
	本次作业我所使用的编程语言为Python \cite{python},编辑环境以jupyter notebook \cite{notebook} 为主。作业中我使用的库包见表 \ref{tab:import}。
	
	\begin{table}[H]
		\renewcommand\arraystretch{1.35}
		\caption{本作业中使用的库包}
		\label{tab:import}
		\centering
		
		\begin{tabular}{c|c}
			\centering
			库包名 &  用途 \\
			\hline
	
			numpy \cite{numpy} & 处理数据,数值计算 \\
			pandas \cite{pandas} & 读取数据,绘图,处理数据 \\
			matplotlib \cite{matplotlib} & 绘图 \\
			scipy \cite{scipy} & 数值计算 \\
			scikit-learn \cite{sklearn} & 分类模型的构造和运算 \\
	
		\end{tabular}
	\end{table}
	
	\subsection{Adult数据集特征分布补充资料}
	\label{apd:dis_detail}
	
	\subsubsection{Adult数据集所有属性分布直方图}
	\label{apd:attri}
	
	\vspace{-0.02\linewidth}
	\begin{figure}[H]
		\centering
		\includegraphics[width=0.97\linewidth]{img/all_features.png}
		\caption{Adult数据集所有属性的分布直方图}
		\label{fig:all_features}
	\end{figure}
	
	\subsubsection{native-country特征的详细分布}
	\label{apd:native-country}
	
	\vspace{-0.02\linewidth}
	\begin{figure}[H]
		\centering
		\includegraphics[width=0.80\linewidth]{img/native_country_dis2.pdf}
		\caption{Adult数据集native-country特征分布(除United-States)}
		\label{fig:class_feature_dis_detail}
	\end{figure}
	
	\begin{figure}[H]
		\centering
		\subfigure[hours-per-week]{
			\includegraphics[width=0.45\linewidth]{img/hours_per_week_box.pdf}
		}
		\subfigure[education-num]{
			\includegraphics[width=0.45\linewidth]{img/edu_num_box.pdf}
		}
		\caption{hours-per-week和education-num特征箱线图}
		\label{fig:box}
	\end{figure}
	
	\subsubsection{Adult数据集特征向量化后的相关系数热度图(含相关系数,电子版可放大后观看)}
	\label{apd:heat}
	
	%\vspace{-0.02\linewidth}
	\begin{figure}[H]
		\centering
		\includegraphics[width=0.85\linewidth]{img/cof_heat2_anno.png}
		\caption{Adult数据集特征向量化后的相关系数热度图(含相关系数)}
		\label{fig:heat4}
	\end{figure}
	
	\subsubsection{各类别型特征下的详细类名}
	\label{apd:classes}
	
	\begin{enumerate}
	
	\item \textbf{workclass} Private, Local-gov, Self-emp-not-inc, Federal-gov, State-gov, Self-emp-inc, Without-pay, Never-worked;
	
	\item \textbf{education} 11th, HS-grad, Assoc-acdm, Some-college, 10th, Prof-school, 7th-8th, Bachelors, Masters, Doctorate, 5th-6th, Assoc-voc, 9th, 12th, 1st-4th, Preschool;
	
	\item \textbf{marital-status} Never-married, Married-civ-spouse, Widowed, Divorced, Separated, Married-spouse-absent, Married-AF-spouse;
	
	\item \textbf{occupation} Machine-op-inspct, Farming-fishing, Protective-serv, Other-service, Prof-specialty, Craft-repair, Adm-clerical, Exec-managerial, Tech-support, Sales, Priv-house-serv, Transport-moving, Handlers-cleaners, Armed-Forces;
	
	\item \textbf{relationship} Own-child, Husband, Not-in-family, Unmarried, Wife, Other-relative;
	
	\item \textbf{race} Black, White, Asian-Pac-Islander, Other, Amer-Indian-Eskimo;
	
	\item \textbf{sex} Male, Female;
	
	\item \textbf{native-country} United-States, Cuba, Jamaica, India, Mexico, Puerto-Rico, Honduras, England, Canada, Germany, Iran, Philippines, Poland, Columbia, Cambodia, Thailand, Ecuador, Laos, Taiwan, Haiti, Portugal, Dominican-Republic, El-Salvador, France, Guatemala, Italy, China, South, Japan, Yugoslavia, Peru, Outlying-US(Guam-USVI-etc), Scotland, Trinadad\&Tobago, Greece, Nicaragua, Vietnam, Hong, Ireland, Hungary, Holand-Netherlands.
	       
	\end{enumerate}
	
\end{appendix}

\bibliographystyle{ieeetr}
\bibliography{bio}

%========================================================================
\end{document}